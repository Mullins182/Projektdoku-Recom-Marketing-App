% \vspace{7pt}
% \begin{center}
% \begin{tikzpicture}[
%   font=\sffamily\itshape\fontsize{13}{15}\selectfont,
%   actor/.style={draw, minimum size=1cm},
%   include/.style={dashed,->,>=latex}
% ]

%     % Funktion zum Zeichnen eines Strichmännchens
%     \newcommand{\actor}[3]{%
%         % Kopf
%         \draw (#1,#2) circle (0.3);
%         % Körper
%         \draw (#1,#2-0.3) -- (#1,#2-1);
%         % Arme
%         \draw (#1-0.5,#2-0.6) -- (#1+0.5,#2-0.6);
%         % Beine
%         \draw (#1,#2-1) -- (#1-0.5,#2-1.5);
%         \draw (#1,#2-1) -- (#1+0.5,#2-1.5);
%         % Name
%         \node[below=1.7cm] at (#1,#2) {#3};
%     }

%     % Systemgrenze
%     \draw[thick, fill=yellow!10]
%         (-7,-11) rectangle (2.75,4)
%         node[above,xshift=-5cm, yshift=-0.8cm, fill=yellow!25] {Empfehlungsmarketing-App};

%     % Akteure (Strichmännchen)
%     \actor{-8.5}{2}{Werber}
%     \actor{-8.5}{-2}{Geworbener}
%     \actor{4.5}{2}{Mitarbeiter}

%     % Use Cases
%     % \node[draw, ellipse, fill=yellow!50, minimum height=1cm, minimum width=2.5cm] (ucPIN) at (-2,2) {PIN validieren};
%     \node[draw, ellipse, fill=yellow!50, minimum height=1cm, minimum width=2.5cm,
%     text width=5cm, align=center] (ucLogin) at (-2,2) {Einloggen in Mitarbeiterbereich};
%     \node[draw, ellipse, fill=yellow!50, minimum height=1cm, minimum width=2.5cm,
%     text width=5cm, align=center] (ucRedeem) at (-2,-2) {Gutschein scannen / prüfen / entwerten};
%     \node[draw, ellipse, fill=yellow!50, minimum height=1cm, minimum width=2.5cm,
%     text width=5cm, align=center] (ucRegister) at (-2,-4.0) {Registrieren mit Email-Adresse};
%     \node[draw, ellipse, fill=yellow!50, minimum height=1cm, minimum width=2.5cm, text width=5cm, align=center] (ucSendMail) at (-2,-8.0) {Affiliate-Link und/oder Gutschein senden};

%     % Verbindungen zu Akteuren
%     \draw (-7.5,1.0) -- (ucRegister.west);
%     \draw (-7.5,-3.0) -- (ucRegister.west);
%     \draw (3.5,1.4) -- (ucRedeem.east);
%     \draw (3.5,1.4) -- (ucLogin.east);

%     % Gutschein entwerten <<include>> Einloggen  (Pflicht-Abhängigkeit)
%     \draw[include] (ucRedeem.north) --
%     node[right,sloped,above]{\fontsize{12}{14}\selectfont\bfseries <<include>>}
%     (ucLogin.south);
%     \draw[include] (ucRegister.south) --
%     node[right,sloped,above]{\fontsize{12}{14}\selectfont\bfseries <<include>>}
%     (ucSendMail.north);

% \end{tikzpicture}
% \end{center}

\vspace{7pt}
\begin{center}
\begin{tikzpicture}[
  font=\sffamily\itshape\fontsize{13}{15}\selectfont,
  actor/.style={draw, minimum size=1cm},
  include/.style={dashed,->,>=Latex, line width=1pt},
  extend/.style={dashed,->,>=Latex, line width=1pt},
  note/.style={
    draw,
    fill=yellow!20,
    font=\scriptsize,
    align=left,
    inner sep=3pt,
    rectangle,
    minimum width=2.7cm,
    minimum height=1.5cm,
    path picture={
      % gefaltete Ecke oben links (IHK-konform)
      \draw (path picture bounding box.north west) ++(0.4,0)
        -- ++(0,-0.4) -- (path picture bounding box.north west) -- cycle;
    }
  }
]

    % Funktion zum Zeichnen eines Strichmännchens
    \newcommand{\actor}[3]{%
        % Kopf
        \draw (#1,#2) circle (0.3);
        % Körper
        \draw (#1,#2-0.3) -- (#1,#2-1);
        % Arme
        \draw (#1-0.5,#2-0.6) -- (#1+0.5,#2-0.6);
        % Beine
        \draw (#1,#2-1) -- (#1-0.5,#2-1.5);
        \draw (#1,#2-1) -- (#1+0.5,#2-1.5);
        % Name
        \node[below=1.7cm] at (#1,#2) {#3};
    }

    % Systemgrenze
    \draw[thick, fill=yellow!10]
        (-7,-11) rectangle (3,5)
        node[above,xshift=-5cm, yshift=-0.8cm, fill=yellow!25] {Empfehlungsmarketing-App};

    % Akteure
    \actor{-8.5}{3}{Werber}
    \actor{-8.5}{0}{Geworbener}
    \actor{4.5}{3}{Mitarbeiter}

    % Use Cases
    \node[draw, ellipse, fill=yellow!50, minimum height=1cm, minimum width=3cm,
    text width=5cm, align=center] (ucLogin) at (-2,3.0) {Einloggen in Mitarbeiterbereich};

    \node[draw, ellipse, fill=yellow!50, minimum height=1cm, minimum width=3cm,
    text width=5cm, align=center] (ucRedeem) at (-2,-1.0) {Gutschein scannen / prüfen / entwerten};

    \node[draw, ellipse, fill=yellow!50, minimum height=1cm, minimum width=3cm,
    text width=2.5cm, align=center] (ucAffLink) at (-4.5,-5.0) {Affiliate-Link senden};

    \node[draw, ellipse, fill=yellow!50, minimum height=1cm, minimum width=3cm,
    text width=2cm, align=center] (ucVoucherSend) at (0.5,-5.0) {Gutschein senden};

    \node[draw, ellipse, fill=yellow!50, minimum height=1cm, minimum width=3cm,
    text width=5cm, align=center] (ucRegister) at (-2,-9.0) {Registrieren mit Email-Adresse};

    % Verbindungen zu Akteuren
    \draw (-7.5,2.0) -- (ucRegister.west);
    \draw (-7.5,-0.5) -- (ucRegister.west);
    \draw (3.5,2.8) -- (ucRedeem.east);
    \draw (3.5,3.2) -- (ucLogin.east);

    % <<include>>: Gutscheinprüfung benötigt Login
    \draw[include] (ucRedeem.north) --
      node[right,sloped,above]{\fontsize{11}{13}\selectfont\bfseries <<include>>}
      (ucLogin.south);

    % <<include>>: Registrierung umfasst immer Affiliate-Link und Gutschein senden
    \draw[include] (ucAffLink.south) --
      node[right,sloped,above]{\fontsize{11}{13}\selectfont\bfseries <<include>>}
      (ucRegister.north);
    \draw[include] (ucVoucherSend.south) --
      node[left,sloped,above]{\fontsize{11}{13}\selectfont\bfseries <<include>>}
      (ucRegister.north);

    % <<extend>>: Gutschein einlösen -> Werber erhält Gutschein
    \draw[extend] (ucVoucherSend.north east) -- (ucRedeem.south east)
      node[midway,sloped,above]{\fontsize{11}{13}\selectfont\bfseries <<extend>>};

    % UML-Notiz (IHK-konform, Ecke oben links), an die Beziehung gehängt
    \node[note] (note1) at (-2,-3.0) {Bedingung:\\ Gutschein entwertet};
    \draw[dashed] (note1.west) -- ++(-0.5,0) |- ($(ucVoucherSend.north east)!0.15!(ucRedeem.south east)$);

    % ===== Legende unten rechts =====
    \node[draw, fill=yellow!20, align=left, font=\scriptsize, anchor=north east] at (3,-9) (legend) {
        \textbf{Legende}\\[3pt]
        \begin{tabular}{@{}ll@{}}
          \raisebox{0.5ex}{\tikz{\draw[dashed,->,>=Latex,line width=1pt] (0,0)--(0.8,0);}} & <<include>> = verpflichtender Use Case \\[6pt]
          \raisebox{0.5ex}{\tikz{\draw[dashed,->,>=Latex,line width=1pt] (0,0)--(0.8,0);}} & <<extend>> = optionaler Use Case \\
        \end{tabular}
    };

\end{tikzpicture}
\end{center}
