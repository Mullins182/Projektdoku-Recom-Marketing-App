
\vspace{7pt}
\begin{center}
  \begin{tikzpicture}[
      font=\sffamily\itshape\fontsize{13}{15}\selectfont,
      actor/.style={draw, minimum size=1cm},
      include/.style={dashed,->,>=Latex, line width=1pt},
      extend/.style={dashed,->,>=Latex, line width=1pt},
      note/.style={
          draw,
          fill=yellow!40,
          font=\scriptsize,
          align=left,
          inner sep=3pt,
          rectangle,
          minimum width=2.7cm,
          minimum height=1.7cm,
          path picture={
              % gefaltete Ecke oben links (IHK-konform)
              \draw (path picture bounding box.north west) ++(0.5,0)
              -- ++(-0.5,-0.4) -- (path picture bounding box.north west) -- cycle;
            }
        }
    ]

    % Funktion zum Zeichnen eines Strichmännchens
    \newcommand{\actor}[3]{%
      % Kopf
      \draw (#1,#2) circle (0.3);
      % Körper
      \draw (#1,#2-0.3) -- (#1,#2-1);
      % Arme
      \draw (#1-0.5,#2-0.6) -- (#1+0.5,#2-0.6);
      % Beine
      \draw (#1,#2-1) -- (#1-0.5,#2-1.5);
      \draw (#1,#2-1) -- (#1+0.5,#2-1.5);
      % Name
      \node[below=1.7cm] at (#1,#2) {#3};
    }

    % Systemgrenze
    \draw[thick, fill=yellow!15]
    (-7,-12.1) rectangle (3,5)
    node[above,xshift=-5cm, yshift=-0.8cm, fill=yellow!33] {Empfehlungsmarketing-App};

    % Akteure
    \actor{-8.5}{-6}{Werber}
    \actor{-8.5}{-9}{Geworbener}
    \actor{4.5}{3}{Mitarbeiter}

    % Use Cases
    \node[draw, ellipse, fill=orange!45, minimum height=1cm, minimum width=3cm,
      text width=5cm, align=center] (ucLogin) at (-2,3.0) {Einloggen in Mitarbeiterbereich};

    \node[draw, ellipse, fill=orange!45, minimum height=1cm, minimum width=1cm,
      text width=3cm, align=center] (ucScan) at (-4.5,0.0) {Gutschein scannen};

    \node[draw, ellipse, fill=orange!45, minimum height=1cm, minimum width=1cm,
      text width=3cm, align=center] (ucRedeem) at (0.5,0.0) {Gutschein entwerten};

    \node[draw, ellipse, fill=orange!45, minimum height=1cm, minimum width=3cm,
      text width=2.5cm, align=center] (ucAffLink) at (-4.7,-6.5) {Affiliate-Link senden};

    \node[draw, ellipse, fill=orange!45, minimum height=1cm, minimum width=3cm,
      text width=2cm, align=center] (ucVoucherSend) at (1.0,-6.5) {Gutschein senden};

    \node[draw, ellipse, fill=orange!45, minimum height=1cm, minimum width=3cm,
      text width=5cm, align=center] (ucRegister) at (-2,-9.4) {Registrieren mit Email-Adresse};

    % Verbindungen zu Akteuren
    \draw (-7.5,-7.0) -- (ucRegister.west);
    \draw (-7.5,-9.5) -- (ucRegister.west);
    \draw (4,2.2) -- (ucRedeem.east);
    \draw (4,2.2) -- (ucScan.east);
    \draw (4,2.2) -- (ucLogin.east);

    % <<include>>: Gutscheinprüfung benötigt Login
    \draw[include] (ucRedeem.north) --
    node[right,sloped,above]{\fontsize{11}{13}\selectfont\bfseries <<include>>}
    (ucLogin.south);
    \draw[include] (ucScan.north) --
    node[right,sloped,above]{\fontsize{11}{13}\selectfont\bfseries <<include>>}
    (ucLogin.south);

    % <<include>>: Registrierung umfasst immer Affiliate-Link und optional Gutschein senden
    \draw[include] (ucRegister.north) --
    node[right,sloped,above]{\fontsize{11}{13}\selectfont\bfseries <<include>>}
    (ucAffLink.south);
    \draw[extend] (ucVoucherSend.south) --
    node[left,sloped,above]{\fontsize{11}{13}\selectfont\bfseries <<extend>>}
    (ucRegister.north);

    % <<extend>>: Gutschein einlösen -> Werber erhält Gutschein
    \draw[extend] (ucVoucherSend.north) -- (ucRedeem.south)
    node[midway,sloped,above]{\fontsize{11}{13}\selectfont\bfseries <<extend>>};

    % UML-Notiz (IHK-konform, Ecke oben links), an die Beziehung gehängt
    \node[note] (note1) at (-4,-2.0) {Bedingung:\\ Gutschein wird entwertet};
    \draw[dashed] (note1.east) -- ++(0,0) |- ($(ucVoucherSend.north)!0.75!(ucRedeem.south)$);

    \node[note] (note2) at (-2,-4.5) {Bedingung:\\ Registrierung über\\ Affiliate-Link};
    \draw[dashed] (note2.south) -- ++(0,0) |- ($(ucVoucherSend.south)!0.05!(ucRegister.north)$);

    % ===== Legende unten rechts =====
    \node[draw, fill=yellow!20, align=left, font=\scriptsize, anchor=north east] at (3,-10.3) (legend) {
      \textbf{Legende}\\[3pt]
      \begin{tabular}{@{}ll@{}}
        \raisebox{0.5ex}{\tikz{\draw[dashed,->,>=Latex,line width=1pt] (0,0)--(0.8,0);}} & <<include>> = verpflichtender Use Case \\[6pt]
        \raisebox{0.5ex}{\tikz{\draw[dashed,->,>=Latex,line width=1pt] (0,0)--(0.8,0);}} & <<extend>> = optionaler Use Case       \\
      \end{tabular}
    };

  \end{tikzpicture}
\end{center}
