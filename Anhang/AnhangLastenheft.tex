\subsection{Lastenheft (Auszug)}
\label{app:Lastenheft}
Es folgt ein Auszug aus dem Lastenheft mit Fokus auf die Anforderungen:

Die Anwendung muss folgende Anforderungen erfüllen: 
\begin{enumerate}[itemsep=0em,partopsep=0em,parsep=0em,topsep=0em]
\item Benutzeroberfläche (\ac{GUI})
	\begin{enumerate}
	\item Die Webanwendung muss über ein \ac{GUI} verfügen, über das sich sowohl 
    Kunden, als auch Interessenten mit ihrer Email-Adresse
    registrieren können.
	\item Das Design des \ac{GUI} soll schlicht und modern gehalten werden.
        \item Außerdem soll es einen Mitarbeiterbereich mit Login geben,
        in dem Mitarbeiter die Möglichkeit haben den \ac{QR}-Code von Gutscheinen
        zu scannen und die Gültigkeit zu überprüfen, sowie den Gutschein zu entwerten.
	\end{enumerate}
\item Responsives Design
	\begin{enumerate}
	\item Die Webanwendung soll in allen gängigen Browsern funktionieren.
        \item Sie soll auf Monitoren aller Größen über den PC nutzbar sein.
        \item Außerdem soll sie über mobile Geräte nutzbar sein.
	\end{enumerate}
\item Sicherheit und Datenschutz
	\begin{enumerate}
        \item Die Kundendaten (Email-Adresse) sollen in verschlüsselter Form
        in der Datenbank gespeichert werden, um sensible Daten zusätzlich zu 
        schützen
        \item Die Datenübertragungen sollen mit \ac{SSL/TLS} gesichert sein.
        \item Die \ac{DSGVO} Konformität soll gewährleistet sein.
	\end{enumerate}
\item Automatische Affiliate-Link und Gutscheinerzeugung und Versand (Backend)
        \begin{enumerate}
            \item Bei Registrierung eines Kunden per Email-Adresse soll dieser 
            automatisch einen Affiliate-Link zugesandt bekommen, welcher mit seiner
            user\_id in der Datenbank assoziiert ist.
            \item Registriert sich ein Interessent durch Aufruf des Affiliate-Links eines Kunden, soll dem Interessenten ein Affiliate-Link und ein Gutschein
            im \ac{PDF}-Format, inklusive \ac{QR}-Code zum Einlösen des Gutscheins, 
            automatisch zugesandt werden. 
            \item Wird der Gutschein eines Interessenten durch einen Mitarbeiter durch Scannen des \ac{QR}-Codes auf dem Gutschein eingelöst, soll dem Kunden (Werber) ebenfalls ein Gutschein im \ac{PDF}-Format erzeugt und zugesandt werden.
        \end{enumerate}
\end{enumerate}

