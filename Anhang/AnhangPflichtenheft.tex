\subsection{Pflichtenheft (Auszug)}
\label{app:Pflichtenheft}
Es folgt ein Auszug aus dem Pflichtenheft mit Fokus auf die konkret umzusetzenden Anforderungen:

\begin{enumerate}[itemsep=0em,partopsep=0em,parsep=0em,topsep=0em]
    \item Musskriterien
    \begin{enumerate}
        \item Empfehlungslinks
        \begin{itemize}
            \item Werber können personalisierte Empfehlungslinks generieren.
            \item Empfehlungslinks enthalten einen eindeutigen Token und einen QR-Code.
            \item Geworbene registrieren sich über diesen Link und werden automatisch dem Werber zugeordnet.
        \end{itemize}

        \item Gutscheinsystem
        \begin{itemize}
            \item Ein Mitarbeiter überprüft die Gültigkeit des Gutscheins per App-Scan.
            \item Nach erfolgreicher Prüfung wird der Gutschein entwertet.
            \item Beim Kauf bestimmter Produkte unter Vorlage des Gutscheins erhält der Werber \bzw der Geworbene einen Rabatt in Höhe von \SI{10}{\percent}.
        \end{itemize}

        \item Benutzeroberfläche
        \begin{itemize}
            \item Eine Weboberfläche ermöglicht die Registrierung für Werber und Geworbene.
            \item Ein Mitarbeiterbereich erlaubt nach Login mit einem Pincode das Scannen und Entwerten (nach erfolgreicher automatischer Validierung) von Gutscheinen.
            \item Die Oberfläche soll schlicht, modern und responsiv gestaltet sein.
        \end{itemize}

        \item Datenverwaltung
        \begin{itemize}
            \item Speicherung und Verwaltung der generierten Links und Gutscheincodes in einer Datenbank.
            \item Historisierung relevanter Daten zur Nachvollziehbarkeit.
            \item Absicherung der erfassten personenbezogenen Daten (E-Mail-Adressen).
        \end{itemize}
    \end{enumerate}

    \item Rahmenbedingungen und Abgrenzung
    \begin{enumerate}
        \item Rechtliche Anforderungen an Datenschutz (DSGVO) sind einzuhalten.
        \item Die Lösung soll innerhalb des vorgegebenen Projektzeitraums implementiert und dokumentiert werden.
        \item Es sind nur Standard-Webtechnologien erlaubt, externe Lizenzkosten sollen vermieden werden.
    \end{enumerate}
\end{enumerate}
