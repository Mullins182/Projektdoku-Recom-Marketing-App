% !TEX root = ../Projektdokumentation.tex

% Seitenränder -----------------------------------------------------------------
\setlength{\topskip}{\ht\strutbox} % behebt Warnung von geometry
\geometry{
  a4paper,
  left=25mm,        % linker Rand
  right=25mm,       % rechter Rand
  top=7mm,          % oberer Rand inkl. Kopfzeile -> Kopf rückt höher
  bottom=25mm,      % unterer Rand
  includehead,      % Kopfbereich in die Berechnung einbeziehen
  includefoot,      % Fußbereich in die Berechnung einbeziehen
  headsep=8mm,      % Abstand Kopfzeile -> Textblock
  footskip=12mm     % Abstand Textblock -> Fußzeile
}

\makeatletter
\makeatother


% Kopf-/Fußzeilen mit KOMA-Script ---------------------------------------------
\usepackage[
  automark,   % Kapitel-/Abschnittsangaben in Kopfzeile automatisch erstellen
  headsepline,% Trennlinie unter der Kopfzeile
  ilines      % Trennlinie linksbündig ausrichten
]{scrlayer-scrpage}

% --- EINSTELLUNG: Höhe des Logos im Header (einfach hier ändern) -------------
\newlength{\HeaderLogoHeight}
\setlength{\HeaderLogoHeight}{23mm} % <- Höhe des Logos in der Kopfzeile

% Kopf- und Fußzeilen ----------------------------------------------------------
\clearpairofpagestyles
\pagestyle{scrheadings}

% Kopfzeile
\renewcommand{\headfont}{\normalfont} % Schriftform der Kopfzeile
\ihead{%
  \large\textsc{\titel}\\
  \small\untertitel \\[2ex]
  \textit{\headmark}%
}
\chead{}
\ohead{\includegraphics[height=\HeaderLogoHeight]{\betriebLogo}}

% Headheight passend zum Logo und Text (ohne geometry anzufassen)
\setlength{\headheight}{\dimexpr \HeaderLogoHeight + 12pt\relax}

% Fußzeile
\ifoot*{\autorName}
\cfoot*{\pagemark}   % <-- Seitenzahl jetzt zentriert
\ofoot*{}            % rechts leer

% Überschriften nach DIN 5008 in einer Fluchtlinie -----------------------------

% Abstand zwischen Nummerierung und Überschrift definieren
\newcommand{\headingSpace}{1.5cm}

% Abschnittsüberschriften im selben Stil wie beim Inhaltsverzeichnis einrücken
\renewcommand*{\othersectionlevelsformat}[3]{%
  \makebox[\headingSpace][l]{#3\autodot}%
}

% Für die Einrückung wird das Paket tocloft benötigt
\cftsetindents{section}{0.0cm}{\headingSpace}
\cftsetindents{subsection}{0.0cm}{\headingSpace}
\cftsetindents{subsubsection}{0.0cm}{\headingSpace}
\cftsetindents{figure}{0.0cm}{\headingSpace}
\cftsetindents{table}{0.0cm}{\headingSpace}

% Allgemeines ------------------------------------------------------------------
\onehalfspacing % Zeilenabstand 1,5 Zeilen
\frenchspacing  % erzeugt ein wenig mehr Platz hinter einem Punkt

% Schusterjungen und Hurenkinder vermeiden
\clubpenalty = 10000
\widowpenalty = 10000
\displaywidowpenalty = 10000

% Quellcode-Ausgabe formatieren
\lstset{numbers=left, numberstyle=\tiny, numbersep=5pt, breaklines=true}
\lstset{emph={square}, emphstyle=\color{red}, emph={[2]root,base}, emphstyle={[2]\color{blue}}}

\counterwithout{footnote}{section} % Fußnoten fortlaufend durchnummerieren
\setcounter{tocdepth}{3}           % bis subsubsection ins Inhaltsverzeichnis
\setcounter{secnumdepth}{3}        % bis subsubsection nummerieren

% Aufzählungen anpassen
\renewcommand{\labelenumi}{\arabic{enumi}.}
\renewcommand{\labelenumii}{\arabic{enumi}.\arabic{enumii}.}
\renewcommand{\labelenumiii}{\arabic{enumi}.\arabic{enumii}.\arabic{enumiii}}

% Tabellenfärbung:
\definecolor{heading}{rgb}{0.64,0.78,0.86}
\definecolor{odd}{rgb}{0.9,0.9,0.9}

% Seitenstil für Verzeichnisse: Kopf behalten, Fuß leer
\newpairofpagestyles{verzeichnis}{
  \ihead{%
    \large\textsc{\titel}\\
    \small\untertitel \\[2ex]
    \textit{\headmark}%
  }
  \chead{}
  \ohead{\includegraphics[height=\HeaderLogoHeight]{\betriebLogo}}
  % Fußzeile komplett leer:
  \ifoot{}
  \cfoot{}
  \ofoot{}
}
