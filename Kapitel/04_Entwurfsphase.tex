% !TEX root = ../Projektdokumentation.tex
\section{Entwurfsphase}
\label{sec:Entwurfsphase}

\subsection{Architektur}

\subsubsection{Schichtenmodell}
Die Anwendung ist in drei Schichten gegliedert:

\begin{itemize}
  \item \textbf{Präsentationsschicht (Frontend):}
        \renewcommand{\labelitemii}{$\rightarrow$} % Nur für die innere Ebene
        \begin{itemize}
          \item Darstellung der Benutzeroberfläche und Entgegennahme von Eingaben.
          \item Umsetzung über folgende statische Seiten und Skripte:
                \begin{itemize}
                  \item (\texttt{public/index.html}
                  \item \texttt{public/employee.html}
                  \item \texttt{public/redeem.html}
                  \item CSS in \texttt{public/css/styles.css}
                  \item JavaScript in \texttt{public/js/app.js} und \texttt{public/js/employee.js})
                \end{itemize}
          \item Die Kommunikation erfolgt ausschließlich per \texttt{fetch-API}\quelle{4} (\ac{JSON}) mit der Anwendungsschicht unter \texttt{php/api.php}.
          \item Für die QR-Code-Erkennung in der Mitarbeiteransicht wird die ZXing-Library eingebunden.
        \end{itemize}

  \item \textbf{Anwendungsschicht (Controller/Services):}
        \renewcommand{\labelitemii}{$\rightarrow$} % Nur für die innere Ebene
        \begin{itemize}
          \item Geschäftslogik und Ablaufsteuerung.
          \item\texttt{php/api.php} dient als Controller und bietet \ac{JSON}-basierte Endpunkte (\zB \textit{register}, \textit{validate\_voucher}, \textit{redeem\_voucher}, \textit{employee\_login}).
          \item Die Logik ist in Service-Klassen gekapselt. Dazu gehören:
                \begin{itemize}
                  \item \texttt{php/Auth.php} (Authentifizierung)
                  \item \texttt{php/Csrf.php} (\ac{CSRF}-Token)
                  \item \texttt{php/Voucher.php} (Gutschein-Erstellung \inkl QR/PDF)
                  \item \texttt{php/Mailer.php} (E-Mail-Versand und Protokollierung)
                  \item \texttt{php/Crypto.php} (\ac{AES}-256-GCM-Verschlüsselung)
                  \item \texttt{php/bootstrap.php} (Autoloading, \texttt{.env} Zeitzone).
                \end{itemize}
        \end{itemize}
        \newpage
  \item \textbf{Datenzugriffsschicht (Repository/Persistenz):}
        \renewcommand{\labelitemii}{$\rightarrow$} % Nur für die innere Ebene
        \begin{itemize}
          \item Zugriff auf die MySQL-Datenbank und Persistenz.\\
                Dies erfolgt zentral über \texttt{php/Database.php} (PDO-Initialisierung).
          \item Alle Datenbank-Operationen werden mit Prepared Statements umgesetzt (\zB in \texttt{php/api.php}, \texttt{php/Voucher.php}, \texttt{php/Mailer.php}).
          \item Genutzte Tabellen sind \texttt{users}, \texttt{vouchers},  \texttt{redemptions} und \texttt{mail\_log}.
          \item Sensible Felder werden verschlüsselt gespeichert (\texttt{users.email\_enc}, \texttt{users.email\_iv}, \texttt{users.email\_tag}).
          \item Zur Suche wird ein Hash verwendet (\texttt{users.email\_hash}).
        \end{itemize}
\end{itemize}

\subsubsection{Schichtgrenzen und Kommunikation}
Das Frontend ruft ausschließlich \ac{JSON}-Endpunkte der Anwendungsschicht auf.
Direkte Datenbankzugriffe aus dem Frontend finden nicht statt.
Die Anwendungsschicht greift nur über die Datenzugriffsschicht (\ac{PDO}) auf die Datenbank zu.
Authentifizierung und Sitzungsverwaltung liegen in der Anwendungsschicht.
\ac{CSRF}-Schutz wird über Token realisiert.

\subsubsection{Diagramm}
Zur Visualisierung des Aufbaus dient Abbildung~\ref{fig:architektur_schichten}.

\begin{figure}[H]
  \centering
  \begin{tikzpicture}[node distance=0cm,outer sep=0pt]
    % Präsentation
    \node[draw, fill=blue!15, minimum width=10cm, minimum height=2cm, align=center] (frontend) {Präsentationsschicht\\[0.2cm]
      HTML, CSS, JavaScript, QR-Code-Scanner};

    % Anwendung
    \node[draw, fill=green!15, minimum width=10cm, minimum height=3cm, align=center, below=of frontend, yshift=-0.5cm] (application) {Anwendungsschicht\\[0.2cm]
      \texttt{api.php}, Controller \& Services\\
      (Auth, Voucher, Mailer, Crypto, Csrf)};

    % Daten
    \node[draw, fill=orange!15, minimum width=10cm, minimum height=2cm, align=center, below=of application, yshift=-0.5cm] (database) {Datenzugriffsschicht\\[0.2cm]
      \texttt{Database.php}, MySQL (users, vouchers, redemptions, mail\_log)};

    % Labels
    \node[below=of database, yshift=0.5cm] (label) {};
  \end{tikzpicture}
  \caption{Schichtenarchitektur der Anwendung}
  \label{fig:architektur_schichten}
\end{figure}

\newpage
