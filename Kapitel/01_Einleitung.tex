% !TEX root = ../Projektdokumentation.tex
\section{Einleitung}
\label{sec:Einleitung}

\subsection{Begriffsklärung}
\label{sec:Begriffsklaerung}
Zur Gewährleistung einer konsistenten und verständlichen Verwendung von Begriffen innerhalb dieser Projektdokumentation werden folgende Festlegungen getroffen:
\begin{itemize}
    \item Der \textbf{Kunde}, der seinen Empfehlungslink an Interessenten weiter gibt, wird im weiteren Verlauf als \textbf{Werber} bezeichnet.
    \item Der \textbf{Interessent} wird im weiteren Verlauf als der \textbf{Geworbene} bezeichnet.
\end{itemize}
Durch diese Vereinheitlichung soll die Lesbarkeit und Nachvollziehbarkeit der Dokumentation verbessert werden.

\subsection{Projektumfeld} 
\label{sec:Projektumfeld}
\begin{itemize}
	\item Mein Praktikumsbetrieb ist die Firma ananas.codes e.K., welche im Bereich Webentwicklung und Zeiterfassungssysteme tätig ist. Die Mitarbeiteranzahl beträgt aktuell fünf.
	\item Der Auftraggeber ist die Colibri Contactlinsen und Brillen GmbH, welche aktuell eine Filiale in Lübeck hat und 35 Mitarbeiter beschäftigt.
\end{itemize}


\subsection{Projektziel} 
\label{sec:Projektziel}
\begin{itemize}
	\item Es geht um die Entwicklung einer Empfehlungsmarketing-Webanwendung.
	\item Das Ziel des Projekts war die Konzeption, Entwicklung und Einführung einer webbasierten Empfehlungsmarketing-Applikation für einen realen Kundenbetrieb.
    Die Anwendung soll es ermöglichen:
    \begin{itemize}
    \item Kunden über einen Registrierungsprozess in ein Empfehlungsprogramm aufzunehmen.
    \item Empfehlungslinks inklusive QR-Codes zu generieren und per E-Mail zu versenden.
    \item Gutscheine automatisch auszustellen und Einlösungen im Mitarbeiterbereich vorzunehmen.
    \item Den gesamten Prozess sicher, \acs{DSGVO} konform und plattformunabhängig abzuwickeln.    
    \end{itemize}
Die Entwicklung erfolgte als eigenständiges Projekt im Rahmen der Abschlussprüfung, orientiert an den Phasen des klassischen Projektmanagements (Wasserfallmodell) mit Prozessorientierung.
\end{itemize}

\subsection{Projektbegründung} 
\label{sec:Projektbegruendung}
\begin{itemize}
	\item Gewinnung von Neukunden durch Verteilung von Affiliate-Links durch Werber an Geworbene.
	\item Der Auftraggeber möchte über solch eine moderne Form der Neukundengewinnung verfügen.
\end{itemize}


\subsection{Projektschnittstellen} 
\label{sec:Projektschnittstellen}
Die Anwendung besteht aus einem webbasierten Frontend (\ac{HTML}, \ac{CSS}, JavaScript, Bootstrap) und einem serverseitigen Backend (\ac{PHP}, \ac{Composer}\quelle{5}, \ac{MySQL}).  
Die Kommunikation zwischen Frontend und Backend erfolgt über die \texttt{Fetch-API} mittels \ac{HTTPS}-Requests.  
Daten werden dabei im \ac{JSON}-Format ausgetauscht.  

\begin{itemize}
    \item \textbf{Frontend $\longrightarrow$ Backend:}
    \begin{itemize}
        \item[$\rightarrow$] Über \texttt{fetch()}-Aufrufe werden REST-ähnliche Endpunkte in \texttt{api.php} angesprochen.
        \item[$\rightarrow$] Scanner-Funktion: Im Mitarbeiterbereich wird ein QR-Code-Scanner (\texttt{@zxing/browser} Bibliothek) genutzt, um Gutscheincodes zu erfassen.
    \end{itemize}

    \item \textbf{Backend $\longrightarrow$ Datenbank / interne Systeme:}
    \begin{itemize}
        \item[$\rightarrow$] Speicherung und Validierung von Benutzer- und Gutscheindaten in einer Datenbank.
        \item[$\rightarrow$] Prüfung und Generierung eindeutiger Referral-Codes ohne Kollision.
        \item[$\rightarrow$] Hashing und Fingerprinting von E-Mail-Adressen für Datenschutz.
        \item[$\rightarrow$] Versand von E-Mails (inklusive QR-Code-Anhang) über ein internes Mailer-System.
    \end{itemize}

    \item \textbf{Genehmigung:}
    \begin{itemize}
        \item[$\rightarrow$] Das Projekt wird vom Chef meines Praktikumsbetriebs genehmigt.
    \end{itemize}

    \item \textbf{Benutzer der Anwendung:}
    \begin{itemize}
        \item[$\rightarrow$] Werber, die sich registrieren und ihren Empfehlungslink teilen, um Gutscheine zu erhalten.
        \item[$\rightarrow$] Geworbene, die sich registrieren um ihren Empfehlungslink und einen Gutschein zu erhalten.
        \item[$\rightarrow$] Mitarbeiter, die Gutscheincodes scannen und einlösen, wenn diese erfolgreich validiert wurden.
    \end{itemize}

    \item \textbf{Präsentation der Ergebnisse:}
    \begin{itemize}
        \item[$\rightarrow$] Die Projektergebnisse werden meinem Chef präsentiert.
    \end{itemize}
\end{itemize}

\subsection{Projektabgrenzung} 
\label{sec:Projektabgrenzung}
\begin{itemize}
	\item Meine Projektarbeit umfasst die Entwicklung des Frontend und Backend der Anwendung, sowie das Testen und die abschließende Abnahme durch meinen Chef. Der Rollout beim Kunden ist nicht Bestandteil meiner Projektarbeit.
\end{itemize}
