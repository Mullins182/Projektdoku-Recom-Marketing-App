%-----------------------------------------------------------------------------------------
% Autor dieser Vorlage:
% Stefan Macke (http://fachinformatiker-anwendungsentwicklung.net)
% Permalink zur Vorlage: http://fiae.link/LaTeXVorlageFIAE
%
% Sämtliche verwendeten Abbildungen, Tabellen und Listings stammen von Dirk Grashorn.
%
% Lizenz: Creative Commons 4.0 Namensnennung - Weitergabe unter gleichen Bedingungen
% -----------------------------------------------------------------------------------------

\documentclass[
	ngerman,
	toc=listof, % Abbildungsverzeichnis sowie Tabellenverzeichnis in das Inhaltsverzeichnis aufnehmen
	toc=bibliography, % Literaturverzeichnis in das Inhaltsverzeichnis aufnehmen
	footnotes=multiple, % Trennen von direkt aufeinander folgenden Fußnoten
	parskip=half, % vertikalen Abstand zwischen Absätzen verwenden anstatt horizontale Einrückung von Folgeabsätzen
	numbers=noendperiod % Den letzten Punkt nach einer Nummerierung entfernen (nach DIN 5008)
]{scrartcl}
\usepackage[utf8]{inputenc} % muss als erstes eingebunden werden, da Meta/Packages ggfs. Sonderzeichen enthalten

\usepackage{luatex85}
\pdfminorversion=5 % erlaubt das Einfügen von pdf-Dateien bis Version 1.7, ohne eine Fehlermeldung zu werfen (keine Garantie für fehlerfreies Einbetten!)

% !TEX root = Projektdokumentation.tex

\providecommand{\pruefungstermin}{Winter 2025/2026} % <anpassen>
\providecommand{\ausbildungsberuf}{Fachinformatiker / Anwendungsentwicklung}
\providecommand{\betreff}{Projektdokumentation} % <anpassen>
\providecommand{\betriebLogo}{Bilder/ananas_codes.png} 

\providecommand{\titel}{Empfehlungsmarketing-App}
\providecommand{\untertitel}{Webanwendung zur Empfehlungslinkerstellung und Gutscheingenerierung}
\providecommand{\autorName}{Benjamin Blunk}
\providecommand{\autorAnschrift}{Auf dem Schild 6}
\providecommand{\autorOrt}{23560 Lübeck}
\providecommand{\betriebName}{ananas.codes e.K.}
\providecommand{\betriebAnschrift}{Lindenstraße 8}
\providecommand{\betriebOrt}{23558 Lübeck}
\providecommand{\abgabeOrt}{Lübeck}
\providecommand{\abgabeTermin}{10.12.2025}
 % Metadaten zu diesem Dokument (Autor usw.)
\input{Allgemein/Packages} % verwendete Packages
% !TEX root = ../Projektdokumentation.tex

% Seitenränder -----------------------------------------------------------------
\setlength{\topskip}{\ht\strutbox} % behebt Warnung von geometry
\geometry{
  a4paper,
  left=25mm,        % linker Rand
  right=25mm,       % rechter Rand
  top=7mm,          % oberer Rand inkl. Kopfzeile -> Kopf rückt höher
  bottom=25mm,      % unterer Rand
  includehead,      % Kopfbereich in die Berechnung einbeziehen
  includefoot,      % Fußbereich in die Berechnung einbeziehen
  headsep=8mm,      % Abstand Kopfzeile -> Textblock
  footskip=12mm     % Abstand Textblock -> Fußzeile
}

\makeatletter
\makeatother


% Kopf-/Fußzeilen mit KOMA-Script ---------------------------------------------
\usepackage[
  automark,   % Kapitel-/Abschnittsangaben in Kopfzeile automatisch erstellen
  headsepline,% Trennlinie unter der Kopfzeile
  ilines      % Trennlinie linksbündig ausrichten
]{scrlayer-scrpage}

% --- EINSTELLUNG: Höhe des Logos im Header (einfach hier ändern) -------------
\newlength{\HeaderLogoHeight}
\setlength{\HeaderLogoHeight}{23mm} % <- Höhe des Logos in der Kopfzeile

% Kopf- und Fußzeilen ----------------------------------------------------------
\clearpairofpagestyles
\pagestyle{scrheadings}

% Kopfzeile
\renewcommand{\headfont}{\normalfont} % Schriftform der Kopfzeile
\ihead{%
  \large\textsc{\titel}\\
  \small\untertitel \\[2ex]
  \textit{\headmark}%
}
\chead{}
\ohead{\includegraphics[height=\HeaderLogoHeight]{\betriebLogo}}

% Headheight passend zum Logo und Text (ohne geometry anzufassen)
\setlength{\headheight}{\dimexpr \HeaderLogoHeight + 12pt\relax}

% Fußzeile
\ifoot*{\autorName}
\cfoot*{\pagemark}   % <-- Seitenzahl jetzt zentriert
\ofoot*{}            % rechts leer

% Überschriften nach DIN 5008 in einer Fluchtlinie -----------------------------

% Abstand zwischen Nummerierung und Überschrift definieren
\newcommand{\headingSpace}{1.5cm}

% Abschnittsüberschriften im selben Stil wie beim Inhaltsverzeichnis einrücken
\renewcommand*{\othersectionlevelsformat}[3]{%
  \makebox[\headingSpace][l]{#3\autodot}%
}

% Für die Einrückung wird das Paket tocloft benötigt
\cftsetindents{section}{0.0cm}{\headingSpace}
\cftsetindents{subsection}{0.0cm}{\headingSpace}
\cftsetindents{subsubsection}{0.0cm}{\headingSpace}
\cftsetindents{figure}{0.0cm}{\headingSpace}
\cftsetindents{table}{0.0cm}{\headingSpace}

% Allgemeines ------------------------------------------------------------------
\onehalfspacing % Zeilenabstand 1,5 Zeilen
\frenchspacing  % erzeugt ein wenig mehr Platz hinter einem Punkt

% Schusterjungen und Hurenkinder vermeiden
\clubpenalty = 10000
\widowpenalty = 10000
\displaywidowpenalty = 10000

% Quellcode-Ausgabe formatieren
\lstset{numbers=left, numberstyle=\tiny, numbersep=5pt, breaklines=true}
\lstset{emph={square}, emphstyle=\color{red}, emph={[2]root,base}, emphstyle={[2]\color{blue}}}

\counterwithout{footnote}{section} % Fußnoten fortlaufend durchnummerieren
\setcounter{tocdepth}{3}           % bis subsubsection ins Inhaltsverzeichnis
\setcounter{secnumdepth}{3}        % bis subsubsection nummerieren

% Aufzählungen anpassen
\renewcommand{\labelenumi}{\arabic{enumi}.}
\renewcommand{\labelenumii}{\arabic{enumi}.\arabic{enumii}.}
\renewcommand{\labelenumiii}{\arabic{enumi}.\arabic{enumii}.\arabic{enumiii}}

% Tabellenfärbung:
\definecolor{heading}{rgb}{0.64,0.78,0.86}
\definecolor{odd}{rgb}{0.9,0.9,0.9}

% Seitenstil für Verzeichnisse: Kopf behalten, Fuß leer
\newpairofpagestyles{verzeichnis}{
  \ihead{%
    \large\textsc{\titel}\\
    \small\untertitel \\[2ex]
    \textit{\headmark}%
  }
  \chead{}
  \ohead{\includegraphics[height=\HeaderLogoHeight]{\betriebLogo}}
  % Fußzeile komplett leer:
  \ifoot{}
  \cfoot{}
  \ofoot{}
}
 % Definitionen zum Aussehen der Seiten
\input{Allgemein/Befehle} % eigene allgemeine Befehle, die z.B. die Arbeit mit LaTeX erleichtern
\input{Befehle} % eigene projektspezifische Befehle, z.B. Abkürzungen usw.

\usepackage{float}

% Hilfs-Makros für Attribute
\newcommand{\pk}[1]{\texttt{\textcolor{red!60!black}{#1}}}
\newcommand{\fk}[1]{\texttt{\textcolor{blue!60!black}{#1}}}
\newcommand{\attr}[1]{\texttt{#1}}

% \usepackage{showframe} % Debugging Element-Frames

\begin{document}

\phantomsection
% IHK-Deckblatt (als PDF), ohne Kopf/Fuß:
\pdfbookmark[1]{Eidesstattliche Erklärung}{ihkdeckblatt}
\includepdf[pages=1,pagecommand={\thispagestyle{empty}}]{Bilder/DeckblattIHK}
\clearpage

\phantomsection
\pdfbookmark[1]{Deckblatt}{deckblatt}
% !TEX root = Projektdokumentation.tex

% Temporär: Kopf-/Fußbereiche NICHT berücksichtigen, damit echte Seitenmitte gilt
\newgeometry{
  left=25mm, right=25mm, top=25mm, bottom=25mm,
  includehead=false, includefoot=false
}

\begin{titlepage}
  \pagestyle{empty}    % gesamte Seite ohne Kopf-/Fußzeile
  \thispagestyle{empty}

  \begin{center}
    \vspace*{\fill}
    \begin{minipage}{0.9\textwidth}
      \centering

      \includegraphics[height=18mm]{LogoIHK.pdf}\\[1ex]
      \Large{Abschlussprüfung \pruefungstermin}\\[3ex]

      \Large{\ausbildungsberuf}\\
      \LARGE{\betreff}\\[4ex]

      \huge{\textbf{\titel}}\\[1.5ex]
      \Large{\textbf{\untertitel}}\\[4ex]

      \normalsize
      Abgabetermin: \abgabeOrt, den \abgabeTermin\\[3em]
      \textbf{Prüfungsbewerber:}\\
      \autorName\\
      \autorAnschrift\\
      \autorOrt\\[5ex]

      \textbf{Ausbildungsbetrieb:}\\
      \betriebName\\
      \betriebAnschrift\\
      \betriebOrt\\[5em]

      \includegraphics[height=35mm]{Bilder/ananas_codes.png}\\[2ex]
    \end{minipage}
    \vspace*{\fill}
  \end{center}

  % Titelblatt-Seite wirklich leer lassen (KOMA setzt am Ende gerne 'plain'):
  \thispagestyle{empty}
\end{titlepage}

\restoregeometry

\clearpage

% Inhaltsverzeichnis ----------------------------------------------------------
\phantomsection
\pagenumbering{Roman}
\pdfbookmark[1]{Inhaltsverzeichnis}{inhalt}

\pagestyle{verzeichnis}
\tableofcontents
\clearpage

% Abkürzungsverzeichnis -------------------------------------------------------
\newcommand{\abkvz}{Abkürzungsverzeichnis}
\renewcommand{\nomname}{\abkvz}
\section*{\abkvz}
\markboth{\abkvz}{\abkvz}
\addcontentsline{toc}{section}{\abkvz}

\pagestyle{verzeichnis}
% !TEX root = Projektdokumentation.tex

% Es werden nur die Abkürzungen aufgelistet, die mit \ac definiert und auch benutzt wurden. 
%
% \acro{VERSIS}{Versicherungsinformationssystem\acroextra{ (Bestandsführungssystem)}}
% Ergibt in der Liste: VERSIS Versicherungsinformationssystem (Bestandsführungssystem)
% Im Text aber: \ac{VERSIS} -> Versicherungsinformationssystem (VERSIS)

% Hinweis: allgemein bekannte Abkürzungen wie z.B. bzw. u.a. müssen nicht ins Abkürzungsverzeichnis aufgenommen werden
% Hinweis: allgemein bekannte IT-Begriffe wie Datenbank oder Programmiersprache müssen nicht erläutert werden,
%          aber ggfs. Fachbegriffe aus der Domäne des Prüflings (z.B. Versicherung)

% Die Option (in den eckigen Klammern) enthält das längste Label oder
% einen Platzhalter der die Breite der linken Spalte bestimmt.

\begin{acronym}[WWWWW]
  \acro{AES}{Advanced Encryption Standard: Symmetrisches Verschlüsselungsverfahren.}\acused{AES}
  \acro{API}{Application Programming Interface: Schnittstelle zur Anwendungsprogrammierung.}\acused{API}
  \acro{CDN}{Content Delivery Network: Verteiltes Netz von Servern zur schnelleren Bereitstellung von Inhalten.}\acused{CDN}
  \acro{Cloud}{Cloud Computing: Bereitstellung von IT-Ressourcen über das Internet (z.\,B. Speicher oder Server).}\acused{Cloud}
  \acro{Composer}{Composer: Paket- und Abhängigkeitsverwaltung für PHP-Projekte.}\acused{Composer}
  \acro{CSRF}{Cross-Site Request Forgery: Angriffstechnik zur Ausnutzung von Sitzungen im Web.}\acused{CSRF}
  \acro{CSS}{Cascading Style Sheets: Sprache zur Gestaltung und Formatierung von Webseiten.}\acused{CSS}
  \acro{DOM}{Document Object Model: Standardisierte Schnittstelle zur strukturierten Darstellung von HTML/XML-Dokumenten.}\acused{DOM}
  \acro{DSGVO}{Datenschutz-Grundverordnung: EU-Verordnung zum Schutz personenbezogener Daten.}\acused{DSGVO}
  \acro{EPK}{Ereignisgesteuerte Prozesskette: Methode zur Modellierung von Geschäftsprozessen.}\acused{EPK}
  \acro{ERM}{Entity-Relationship-Modell: Datenmodell zur Darstellung von Entitäten und Beziehungen.}\acused{ERM}
  \acro{FTP}{File Transfer Protocol: Protokoll zur Übertragung von Dateien über ein Netzwerk.}\acused{FTP}
  \acro{Git}{Git: Verteiltes Versionsverwaltungssystem für Quellcode.}\acused{Git}
  \acro{GitHub}{GitHub: Plattform zur Verwaltung von Git-Repositories in der Cloud.}\acused{GitHub}
  \acro{GUI}{Graphical User Interface: Grafische Benutzeroberfläche.}\acused{GUI}
  \acro{HTML}{Hypertext Markup Language: Auszeichnungssprache für Webseiten.}\acused{HTML}
  \acro{HTTPS}{Hypertext Transfer Protocol Secure: Verschlüsseltes Übertragungsprotokoll für sichere Kommunikation im Web.}\acused{HTTPS}
  \acro{MySQL}{MySQL: Relationales Open-Source-Datenbankmanagementsystem (RDBMS).}\acused{MySQL}
  \acro{PDF}{Portable Document Format: Plattformunabhängiges Dateiformat für Dokumente.}\acused{PDF}
  \acro{PDO}{PHP Data Objects: Datenbankschnittstelle in PHP für verschiedene Datenbanken.}\acused{PDO}
  \acro{PHP}{Hypertext Preprocessor: Serverseitige Skriptsprache für Webanwendungen.}\acused{PHP}
  \acro{QR}{Quick Response Code: Zweidimensionaler Barcode zur Codierung von Daten.}\acused{QR}
  \acro{SSL/TLS}{Secure Sockets Layer / Transport Layer Security: Verschlüsselungsprotokolle für sichere Datenübertragung.}\acused{SSL/TLS}
  \acro{SQL}{Structured Query Language: Sprache zur Abfrage und Manipulation relationaler Datenbanken.}\acused{SQL}
  \acro{UI}{User Interface: Benutzerschnittstelle zwischen Mensch und System.}\acused{UI}
  \acro{UML}{Unified Modeling Language: Standardisierte Modellierungssprache für Softwarearchitektur.}\acused{UML}
  \acro{URL}{Uniform Resource Locator: Eindeutige Adresse einer Ressource im Internet.}\acused{URL}
  \acro{VS Code}{Visual Studio Code: Quelloffene Entwicklungsumgebung von Microsoft.}\acused{VS Code}
  \acro{XML}{Extensible Markup Language: Auszeichnungssprache zur strukturierten Speicherung von Daten.}\acused{XML}
  \acro{ZXing}{ZXing („Zebra Crossing“): Open-Source-Bibliothek zur Erkennung und Erstellung von Barcodes und QR-Codes.}\acused{ZXing}
  \acro{@zxing/browser}{@zxing/browser: JavaScript-Bibliothek aus dem ZXing-Projekt zum Scannen von Bar- und QR-Codes im Browser.}\acused{@zxing/browser}
\end{acronym}

\clearpage

% Inhalt ---------------------------------------------------------------------
\clearpage
\pagenumbering{arabic}
\pagestyle{scrheadings}
% !TEX root = Projektdokumentation.tex
% Inhalt.tex – Aggregator für Kapiteldateien
% !TEX root = ../Projektdokumentation.tex
\section{Einleitung}
\label{sec:Einleitung}

\subsection{Begriffsklärung}
\label{sec:Begriffsklaerung}
Zur Gewährleistung einer konsistenten und verständlichen Verwendung von Begriffen innerhalb dieser Projektdokumentation werden folgende Festlegungen getroffen:
\begin{itemize}
    \item Der \textbf{Kunde}, der seinen Empfehlungslink an Interessenten weiter gibt, wird im weiteren Verlauf als \textbf{Werber} bezeichnet.
    \item Der \textbf{Interessent} wird im weiteren Verlauf als der \textbf{Geworbene} bezeichnet.
\end{itemize}
Durch diese Vereinheitlichung soll die Lesbarkeit und Nachvollziehbarkeit der Dokumentation verbessert werden.

\subsection{Projektumfeld} 
\label{sec:Projektumfeld}
\begin{itemize}
	\item Mein Praktikumsbetrieb ist die Firma ananas.codes e.K., welche im Bereich Webentwicklung und Zeiterfassungssysteme tätig ist. Die Mitarbeiteranzahl beträgt aktuell fünf.
	\item Der Auftraggeber ist die Colibri Contactlinsen und Brillen GmbH, welche aktuell eine Filiale in Lübeck hat und 35 Mitarbeiter beschäftigt.
\end{itemize}


\subsection{Projektziel} 
\label{sec:Projektziel}
\begin{itemize}
	\item Es geht um die Entwicklung einer Empfehlungsmarketing-Webanwendung.
	\item Das Ziel des Projekts war die Konzeption, Entwicklung und Einführung einer webbasierten Empfehlungsmarketing-Applikation für einen realen Kundenbetrieb.
    Die Anwendung soll es ermöglichen:
    \begin{itemize}
    \item Kunden über einen Registrierungsprozess in ein Empfehlungsprogramm aufzunehmen.
    \item Empfehlungslinks inklusive QR-Codes zu generieren und per E-Mail zu versenden.
    \item Gutscheine automatisch auszustellen und Einlösungen im Mitarbeiterbereich vorzunehmen.
    \item Den gesamten Prozess sicher, \acs{DSGVO} konform und plattformunabhängig abzuwickeln.    
    \end{itemize}
Die Entwicklung erfolgte als eigenständiges Projekt im Rahmen der Abschlussprüfung, orientiert an den Phasen des klassischen Projektmanagements (Wasserfallmodell) mit Prozessorientierung.
\end{itemize}

\subsection{Projektbegründung} 
\label{sec:Projektbegruendung}
\begin{itemize}
	\item Gewinnung von Neukunden durch Verteilung von Affiliate-Links durch Werber an Geworbene.
	\item Der Auftraggeber möchte über solch eine moderne Form der Neukundengewinnung verfügen.
\end{itemize}


\subsection{Projektschnittstellen} 
\label{sec:Projektschnittstellen}
Die Anwendung besteht aus einem webbasierten Frontend (HTML, CSS, JavaScript, Bootstrap) und einem serverseitigen Backend (PHP, Composer, MySQL).  
Die Kommunikation zwischen Frontend und Backend erfolgt über die \texttt{Fetch-API} mittels \texttt{HTTPS}-Requests.  
Daten werden dabei im \texttt{JSON}-Format ausgetauscht.  

\begin{itemize}
    \item \textbf{Frontend $\longrightarrow$ Backend:}
    \begin{itemize}
        \item[$\rightarrow$] Über \texttt{fetch()}-Aufrufe werden REST-ähnliche Endpunkte in \texttt{api.php} angesprochen.
        \item[$\rightarrow$] Scanner-Funktion: Im Mitarbeiterbereich wird ein QR-Code-Scanner (\texttt{@zxing/browser} Bibliothek) genutzt, um Gutscheincodes zu erfassen.
    \end{itemize}

    \item \textbf{Backend $\longrightarrow$ Datenbank / interne Systeme:}
    \begin{itemize}
        \item[$\rightarrow$] Speicherung und Validierung von Benutzer- und Gutscheindaten in einer Datenbank.
        \item[$\rightarrow$] Prüfung und Generierung eindeutiger Referral-Codes ohne Kollision.
        \item[$\rightarrow$] Hashing und Fingerprinting von E-Mail-Adressen für Datenschutz.
        \item[$\rightarrow$] Versand von E-Mails (inklusive QR-Code-Anhang) über ein internes Mailer-System.
    \end{itemize}

    \item \textbf{Genehmigung:}
    \begin{itemize}
        \item[$\rightarrow$] Das Projekt wird vom Chef meines Praktikumsbetriebs genehmigt.
    \end{itemize}

    \item \textbf{Benutzer der Anwendung:}
    \begin{itemize}
        \item[$\rightarrow$] Werber, die sich registrieren und ihren Empfehlungslink teilen, um Gutscheine zu erhalten.
        \item[$\rightarrow$] Geworbene, die sich registrieren um ihren Empfehlungslink und einen Gutschein zu erhalten.
        \item[$\rightarrow$] Mitarbeiter, die Gutscheincodes scannen und einlösen, wenn diese erfolgreich validiert wurden.
    \end{itemize}

    \item \textbf{Präsentation der Ergebnisse:}
    \begin{itemize}
        \item[$\rightarrow$] Die Projektergebnisse werden meinem Chef präsentiert.
    \end{itemize}
\end{itemize}

\subsection{Projektabgrenzung} 
\label{sec:Projektabgrenzung}
\begin{itemize}
	\item Meine Projektarbeit umfasst die Entwicklung des Frontend und Backend der Anwendung, sowie das Testen und die abschließende Abnahme durch meinen Chef. Der Rollout beim Kunden ist nicht Bestandteil meiner Projektarbeit.
\end{itemize}

\section{Projektplanung}
\label{sec:Projektplanung}

\subsection{Projektphasen}
\label{sec:Projektphasen}
Es wurde das Wasserfallmodell eingesetzt:
\begin{itemize}
  \item \textbf{Analyse}
  \item \textbf{Entwurf}
  \item \textbf{Implementierung}
  \item \textbf{Test}
  \item \textbf{Wartung}
\end{itemize}

\subsection{Abweichungen vom Projektantrag}
\label{sec:AbweichungenProjektantrag}

\begin{itemize}
  \item Detaillierte Projektbeschreibung
        \begin{itemize}
          \item Die Angabe, dass der Geworbene nur einen Rabatt erhält wenn er eine Brille kauft ist nicht richtig. Es wird vom Auftraggeber noch entschieden, für welche Produkte eine Rabattierung möglich ist.
          \item Dem Geworbenen wird ebenfalls ein Affiliate-Link zusätzlich zum Gutschein zugesandt, da dieser dann ebenfalls einen Anreiz hat diesen zu teilen um weitere Gutscheine zu erhalten.
          \item Die Angabe, dass die Gutscheine ihre Gültigkeit nicht verlieren sollen beruht auf einem Missverständnis zwischen mir und meinem Chef. Die Gutscheine sollen nach der Einlösung entwertet werden.
          \item Die Frontend-Library qr-scanner wird ersetzt durch \ac{ZXing} für den QR-Code-Scan zum Einlösen von Gutscheinen, da diese Library häufig genutzt wird in der Softwareentwicklung und sehr zuverlässig und schnell QR-Codes erkennt.
        \end{itemize}
  \item Projektphasen
        \begin{itemize}
          \item Der angegebene geschätzte Zeitbedarf der einzelnen Phasen hat sich nun geändert, aufgrund meiner Neueinschätzung. Die Phase Wartung entfällt, da die Inbetriebnahme der Anwendung beim Kunden nicht Bestandteil meiner Projektarbeit sein soll (Vorgabe meines Praktikumsbetriebs).
        \end{itemize}
\end{itemize}

\subsection{Ressourcen und Werkzeuge}
\begin{itemize}
  \item \textbf{Frontend:} \ac{HTML}5, \ac{CSS}3, JavaScript, Bootstrap 5.3
  \item \textbf{Backend:} \ac{PHP} 8.1, Composer, MySQL
  \item \textbf{Libraries:} vlucas/phpdotenv, \ac{PHP}Mailer (Mailversand), \ac{DOM}\ac{PDF} (PDF-Erzeugung), endroid/qr-code (QR-Codes erzeugen), \ac{ZXing}
  \item \textbf{Datenbank:} My\ac{SQL} (\ac{Cloud}-gehostet)
  \item \textbf{Sicherheit:} PDO, \ac{AES}-256-GCM, HTTPS, \ac{CSRF}-Tokens, Mitarbeiter-Login (PIN)
  \item \textbf{Sonstiges:} \ac{VS Code}, \ac{Git}/\ac{GitHub}, FileZilla \ac{FTP}-Client
\end{itemize}

\subsection{Zeitplanung}
\begin{itemize}
  \item Analyse: 2~h
  \item Entwurf: 16~h
  \item Implementierung: 59~h
  \item Test/Abnahme: 3~h
\end{itemize}
\newpage

\section{Analysephase}
\label{sec:Analysephase}

\subsection{Fachliches Konzept}
Werber erzeugen personalisierte Empfehlungslinks (tokenisiert) inklusive QR-Codes für Geworbene. Geworbene registrieren sich über den Link. Das System erfasst die Email-Adresse und ordnet sie dem Werber zu. Nach dem Kauf bestimmter Produkte durch den Geworbenen, erhält dieser einen bestimmten Rabatt auf sein gekauftes Produkt, nachdem ein Mitarbeiter den QR-Code auf seinem Gutschein mit der App gescannt, validiert und entwertet hat. Dadurch wird automatisch ein Gutschein (PDF) erzeugt und per E-Mail an den Werber versendet.

\subsection{Kosten (Beispiel)}
\begin{itemize}
  \item Entwicklungszeit (80~h) \(\times\) interner Stundensatz (60~€/h) \(\rightarrow\) 4\,800~€,
  \item Hosting/Tools (monatlich), PDF-/Mail-Libraries: Open Source.
\end{itemize}

\subsection{Nutzen}
Erhöhung der Neukundengewinnung über Empfehlungen, transparente Nachverfolgung, minimale manuelle Aufwände durch Automatisierung.

\subsection{Amortisation}
Erwarteter zusätzlicher Deckungsbeitrag durch Neukunden übersteigt die initialen Entwicklungskosten nach \(n\) Monaten.

% !TEX root = ../Projektdokumentation.tex
\section{Entwurfsphase} 
\label{sec:Entwurfsphase}

\subsection{Architektur}

\subsubsection{Schichtenmodell}
Die Anwendung ist in drei Schichten gegliedert:

\paragraph{Präsentationsschicht (Frontend).}
Aufgabe: Darstellung der Benutzeroberfläche und Entgegennahme von Eingaben.  
Umsetzung über statische Seiten und Skripte (\texttt{public/index.html}, \texttt{public/employee.html}, \texttt{public/redeem.html}, CSS in \texttt{public/css/styles.css}, JavaScript in \texttt{public/js/app.js} und \texttt{public/js/employee.js}).  
Die Kommunikation erfolgt ausschließlich per \texttt{fetch} (JSON) mit der Anwendungsschicht unter \texttt{php/api.php}.  
Für die QR-Code-Erkennung in der Mitarbeiteransicht wird die @zxing-Library eingebunden.

\paragraph{Anwendungsschicht (Controller/Services).}
Aufgabe: Geschäftslogik und Ablaufsteuerung.  
\texttt{php/api.php} dient als Controller und bietet JSON-basierte Endpunkte (\zB \textit{register}, \textit{validate\_voucher}, \textit{redeem\_voucher}, \textit{employee\_login}).  
Die Logik ist in Service-Klassen gekapselt. Dazu gehören:  
\texttt{php/Auth.php} (Authentifizierung),  
\texttt{php/Csrf.php} (\ac{CSRF}-Token),  
\texttt{php/Voucher.php} (Gutschein-Erstellung inkl. QR/PDF),  
\texttt{php/Mailer.php} (E-Mail-Versand und Protokollierung),  
\texttt{php/Crypto.php} (\ac{AES}-256-GCM-Verschlüsselung)  
und \texttt{php/bootstrap.php} (Autoloading, \texttt{.env}, Zeitzone).

\paragraph{Datenzugriffsschicht (Repository/Persistenz).}
Aufgabe: Zugriff auf die MySQL-Datenbank und Persistenz.  
Dies erfolgt zentral über \texttt{php/Database.php} (PDO-Initialisierung).  
Alle Datenbank-Operationen werden mit Prepared Statements umgesetzt (\zB in \texttt{php/api.php}, \texttt{php/Voucher.php}, \texttt{php/Mailer.php}).  
Genutzte Tabellen sind \texttt{users}, \texttt{vouchers}, \texttt{redemptions} und \texttt{mail\_log}.  
Sensible Felder werden verschlüsselt gespeichert (\texttt{users.email\_enc}, \texttt{users.email\_iv}, \texttt{users.email\_tag}).  
Zur Suche wird ein Hash verwendet (\texttt{users.email\_hash}).

\subsubsection{Schichtgrenzen und Kommunikation}
Das Frontend ruft ausschließlich JSON-Endpunkte der Anwendungsschicht auf.  
Direkte Datenbankzugriffe aus dem Frontend finden nicht statt.  
Die Anwendungsschicht greift nur über die Datenzugriffsschicht (PDO) auf die Datenbank zu.  
Authentifizierung und Sitzungsverwaltung liegen in der Anwendungsschicht.  
\ac{CSRF}-Schutz wird über Token realisiert.

\subsubsection{Diagramm}
Zur Visualisierung des Aufbaus dient Abbildung~\ref{fig:architektur_schichten}.

\begin{figure}[H]
  \centering
  \begin{tikzpicture}[node distance=0cm,outer sep=0pt]
    % Präsentation
    \node[draw, fill=blue!15, minimum width=10cm, minimum height=2cm, align=center] (frontend) {Präsentationsschicht\\[0.2cm] 
    HTML, CSS, JavaScript, QR-Code-Scanner};
    
    % Anwendung
    \node[draw, fill=green!15, minimum width=10cm, minimum height=3cm, align=center, below=of frontend, yshift=-0.5cm] (application) {Anwendungsschicht\\[0.2cm] 
    \texttt{api.php}, Controller \& Services\\
    (Auth, Voucher, Mailer, Crypto, Csrf)};
    
    % Daten
    \node[draw, fill=orange!15, minimum width=10cm, minimum height=2cm, align=center, below=of application, yshift=-0.5cm] (database) {Datenzugriffsschicht\\[0.2cm] 
    \texttt{Database.php}, MySQL (users, vouchers, redemptions, mail\_log)};
    
    % Labels
    \node[below=of database, yshift=0.5cm] (label) {};
  \end{tikzpicture}
  \caption{Schichtenarchitektur der Anwendung}
  \label{fig:architektur_schichten}
\end{figure}

\newpage

% !TEX root = ../Projektdokumentation.tex
\section{Implementierungsphase} 
\label{sec:Implementierungsphase}

\subsection{Implementierung der Datenstrukturen}
\label{sec:ImplementierungDatenstrukturen}

Zu Beginn der Implementierungsphase habe ich die Datenbankstruktur auf Basis des in der Entwurfsphase erstellten ER-Modells in MySQL umgesetzt.
Die Tabellen für Benutzer, Gutscheine, Einlösungen und das Mail-Log habe ich in einer separaten SQL-Datei (\texttt{schema.sql}) definiert und anschließend per MySQL-Client (phpMyAdmin) importiert.

\lstinputlisting[language=sql, caption={SQL-Definition der zentralen Tabellen}]{Listings/schema.sql}

Die Verbindung zur Datenbank wird über eine zentrale \texttt{Database}-Klasse hergestellt, die \ac{PDO} als Schnittstelle nutzt.
Wichtig war mir, die Verbindung parametrisiert über Umgebungsvariablen (\texttt{.env}-Datei) zu gestalten, um den Host, Port oder Socket, SMTP-Zugangsdaten, Secret \usw flexibel ändern zu können und die sensiblen Daten zu schützen, da ich gleichzeitig eine Serverkonfigurationsdatei (.htaccess) erstellt habe, die den Zugriff auf die Datei mit den Umgebungsvariablen per \ac{URL} verhindert (HTTP 403 - Forbidden Access Error bei Zugriffsversuch).
Ebenso habe ich \ac{PDO} im Exception-Modus aktiviert, um Fehler frühzeitig zu erkennen.

\lstinputlisting[language=php, caption={Konstruktor der Database-Klasse}]{Listings/constructor-db.php}

\subsection{Implementierung der Benutzeroberfläche}
\label{sec:ImplementierungBenutzeroberflaeche}

Die Benutzeroberfläche besteht aus drei HTML-Seiten (\texttt{index.html}, \texttt{employee.html}, \texttt{redeem.html}),
die ich mit Bootstrap und einem eigenen Stylesheet gestaltet habe.
Das Corporate Design ist schlicht, mit klaren Flächen und gut lesbaren Schriften.
Farblich habe ich mich für eine dezente Hintergrundfarbe (\texttt{whitesmoke}) und klare Kontraste bei Buttons und Formularen entschieden.
Durch eigene CSS-Klassen konnte ich \zB die Darstellung des Videoausschnitts im Mitarbeiterbereich optimieren.

\lstinputlisting[language=html, caption={Auszug aus styles.css}]{Listings/styles.css}

Die Interaktivität im Frontend habe ich mit JavaScript umgesetzt. 
Hierzu gehört \zB die Anzeige von Rückmeldungen in einer Bootstrap-Alert-Box.
Für die API-Kommunikation verwende ich die \texttt{fetch}-Funktion mit asynchronen Funktionen.

\lstinputlisting[language=php, caption={Hilfsfunktion zur API-Kommunikation}]{Listings/fetch.js}

\cleardoublepage

\subsection{Implementierung der Geschäftslogik}
\label{sec:ImplementierungGeschaeftslogik}

Die Geschäftslogik habe ich in einzelne PHP-Klassen aufgeteilt, um den Code übersichtlich zu halten.
Ein wichtiger Bestandteil ist die Klasse \texttt{Voucher}, die die Erstellung von Gutscheinen übernimmt.
Dabei wird ein eindeutiger Code generiert, in der Datenbank gespeichert und optional ein Ablaufdatum gesetzt.
Zusätzlich wird aus den Gutschein-Daten direkt ein PDF mit QR-Code erzeugt, das per E-Mail an den Werber \bzw Geworbenen gesendet wird.

\lstinputlisting[language=php, caption={Auszug aus der Voucher-Klasse}]{Listings/voucher.php}

Das QR-Code- und PDF-Rendering erfolgt direkt in PHP über die Bibliotheken \texttt{endroid/qr-code} und \texttt{dompdf/dompdf}.
So konnte ich sicherstellen, dass der komplette Prozess (vom Erstellen bis zum Versenden) serverseitig abläuft.

Screenshots der Anwendung in der Entwicklungsphase mit Dummy-Daten befinden sich im Anhang~\textit{\ref{sec:screenshots}}.

\subsection{Refaktorisieren des Codes}
\label{sec:RefaktorisierenDesCodes}
Nach der ersten Implementierung habe ich den Code mehrfach überarbeitet und verbessert.
Hierzu gehörte das Aufteilen großer Funktionen in kleinere, besser verständliche Einheiten.
Außerdem habe ich wiederkehrende Muster in Hilfsfunktionen ausgelagert, um Redundanzen zu vermeiden.
Beispielsweise habe ich die E-Mail-Versandlogik in eine eigene \texttt{Mailer}-Klasse ausgelagert, die auch das Protokollieren der versendeten Mails übernimmt.
% !TEX root = ../Projektdokumentation.tex
\section{Abnahmephase}
\label{sec:Abnahmephase}

\subsection{Test und Qualitätssicherung}

\subsection{Teststrategie}
\begin{itemize}
  \item Da es sich um ein Ausbildungsprojekt handelt und der Fokus auf der              funktionalen Umsetzung lag,
        habe ich in dieser Projektarbeit ausschließlich manuelle Tests durchgeführt.
        Automatisierte Tests waren nicht Bestandteil des Projektplans.
\end{itemize}

\subsection{Testmethodik}
\label{sec:Testmethodik}
Die Testfälle habe ich aus den in der Planungsphase definierten Use Cases abgeleitet.
Für jeden Use Case habe ich die relevanten Eingaben und erwarteten Ausgaben definiert.
Die Tests habe ich anschließend schrittweise im Browser und über direkte API-Aufrufe geprüft.

\begin{itemize}
  \item \textbf{Registrierung}: Formular mit gültigen und ungültigen E-Mail-Adressen getestet, um Eingabevalidierung zu prüfen.
  \item \textbf{Login}: Sowohl korrekte als auch falsche Zugangsdaten eingegeben, um Fehlermeldungen zu validieren.
  \item \textbf{Gutschein-Erstellung}: Test mit verschiedenen Rabattwerten und Ablaufdaten; anschließend Überprüfung, ob der Gutschein in der Datenbank gespeichert wurde.
  \item \textbf{Gutschein-Einlösung}: QR-Code mit der Mitarbeiteransicht gescannt und geprüft, ob die Einlösung korrekt im Backend protokolliert wird.
  \item \textbf{E-Mail-Versand}: Kontrolle, ob nach erfolgreicher Gutschein-Erstellung eine E-Mail mit PDF-Anhang und QR-Code beim Nutzer ankommt.
\end{itemize}
\clearpage

\subsection{Testergebnisse}
\label{sec:Testergebnisse1}
Alle in der Testphase definierten Use Cases konnten erfolgreich und ohne kritische Fehler durchlaufen werden.
Kleinere Darstellungsprobleme im Firefox wurden durch Anpassung des CSS behoben.
Die Anwendung erfüllt damit die funktionalen Anforderungen aus der Planungsphase.

\begin{center}
  \captionof{table}{Testergebnisse Registrierung und Gutschein-Erstellung}
  \label{tab:testergebnisse1}
  \begin{tabularx}{\textwidth}{|L{1.2cm}|X|L{3.2cm}|L{3.2cm}|L{1.6cm}|}
    \hline
    \textbf{ID} & \textbf{Beschreibung}                 & \textbf{Eingabe/Aktion}                      & \textbf{Erwartetes Ergebnis}                        & \textbf{Status} \\
    \hline
    T-001       & Registrierung mit gültigen Daten      & Gültige E-Mail eingeben, Formular absenden   & Bestätigung im Frontend, QR-Code per Mail           & bestanden       \\
    \hline
    T-002       & Registrierung mit ungültiger E-Mail   & Ungültige E-Mail eingeben, Formular absenden & Fehlermeldung „Bitte gültige E-Mail“                & bestanden       \\
    \hline
    T-003       & Gutschein-Erstellung ohne Ablaufdatum & Rabatt 20\% auswählen, kein Ablaufdatum      & Gutschein gespeichert, PDF mit QR-Code per Mail     & bestanden       \\
    \hline
    T-004       & Gutschein-Erstellung mit Ablaufdatum  & Rabatt 10\% auswählen, Ablaufdatum setzen    & Gutschein gespeichert mit Ablaufdatum, PDF per Mail & bestanden       \\
    \hline
  \end{tabularx}
\end{center}

\newpage

\begin{center}
  \captionof{table}{Testergebnisse Einlösung, Sicherheit und Layout}
  \label{tab:testergebnisse2}
  \begin{tabularx}{\textwidth}{|L{1.2cm}|X|L{3.2cm}|L{3.2cm}|L{1.6cm}|}
    \hline
    \textbf{ID} & \textbf{Beschreibung}                   & \textbf{Eingabe/Aktion}                                     & \textbf{Erwartetes Ergebnis}                                              & \textbf{Status} \\
    \hline
    T-005       & Gutschein-Einlösung durch Mitarbeiter   & QR-Code scannen und bestätigen                              & Eintrag in DB redemptions, Anzeige „eingelöst“                            & bestanden       \\
    \hline
    T-006       & Gutschein-Einlösung mit ungültigem Code & Falschen Code eingeben                                      & Fehlermeldung „Ungültig/abgelaufen“                                       & bestanden       \\
    \hline
    T-007       & E-Mail-Versand bei Gutschein-Erstellung & Gutschein anlegen                                           & Eintrag in DB mail\_log, E-Mail erhalten                                  & bestanden       \\
    \hline
    T-008       & \ac{CSRF}-Schutzprüfung                 & Formular ohne gültigen Token absenden                       & Fehlermeldung im Frontend, kein DB-Eintrag                                & bestanden       \\
    \hline
    T-009       & Responsives Layout                      & Anwendung auf Smartphone öffnen, in allen gängigen Browsern & Inhalte lesbar, Elemente innerhalb des Viewports, QR-Scanner funktioniert & bestanden       \\
    \hline
  \end{tabularx}
\end{center}

\subsection{Zusammenfassung der Testergebnisse}
\label{sec:Testergebnisse2}
Alle definierten Testfälle konnten erfolgreich abgeschlossen werden.
Die Tests haben gezeigt, dass die Anwendung die in der Anforderungsdefinition festgelegten Funktionen zuverlässig umsetzt.
Kleinere Darstellungsprobleme im Firefox wurden während der Testphase behoben, sodass die Anwendung nun in gängigen Browsern ohne Einschränkungen nutzbar ist.

\subsection{Abnahme}
Die Abnahme war erfolgreich und erfolgte durch meinen Chef.
Akzeptanzkriterien: Funktionale Anforderungen, Sicherheit, Nachvollziehbarkeit, PDF-Layout, E-Mail-Templates erfüllt.

\newpage
\section{Fazit}
Das Projektziel, eine funktionsfähige, sichere und benutzerfreundliche Empfehlungsmarketing-App zu erstellen, wurde erreicht. Der Prozess von der Generierung eines Affiliate-Links und/oder Gutscheins bis zum Emailversand an den Kunden, ist automatisiert und nachvollziehbar. Die Architektur ist modular und erweiterbar. Künftige Erweiterungen (\zB erweiterte Gutscheinregeln) sind vorgesehen.
\newpage


% Literatur ------------------------------------------------------------------
\clearpage
\renewcommand{\refname}{Literaturverzeichnis}
\bibliography{Bibliographie}
\bibliographystyle{Allgemein/natdin} % DIN-Stil des Literaturverzeichnisses
\section*{Literatur- und Quellenverzeichnis}
\addcontentsline{toc}{section}{Literatur- und Quellenverzeichnis}
\begin{enumerate}
  \item The PHP Group: PHP Manual. 2025. Verfügbar unter: \url{https://www.php.net/manual/} (Zugriff am 25.09.2025). — Dokumentation zu PHP
  \item The PHP Group: PHP Data Objects (PDO) — Manual. 2025. Verfügbar unter: \url{https://www.php.net/manual/en/book.pdo.php} (Zugriff am 25.09.2025). — PDO-Datenbankschnittstelle
  \item Oracle: MySQL 8.0 Reference Manual. 2025. Verfügbar unter: \url{https://dev.mysql.com/doc/refman/8.0/en/} (Zugriff am 25.09.2025). — MySQL-Dokumentation
  \item vlucas: phpdotenv. 2025. Verfügbar unter: \url{https://github.com/vlucas/phpdotenv} (Zugriff am 25.09.2025). — Umgebungsvariablen in PHP laden
  \item PHPMailer: PHPMailer. 2025. Verfügbar unter: \url{https://github.com/PHPMailer/PHPMailer} (Zugriff am 25.09.2025). — E-Mail-Versand für PHP
  \item dompdf: dompdf. 2025. Verfügbar unter: \url{https://github.com/dompdf/dompdf} (Zugriff am 25.09.2025). — HTML nach PDF rendern
  \item Endroid: endroid/qr-code. 2025. Verfügbar unter: \url{https://github.com/endroid/qr-code} (Zugriff am 25.09.2025). — QR-Codes in PHP erzeugen
  \item ZXing Project: ZXing (“Zebra Crossing”). 2025. Verfügbar unter: \url{https://github.com/zxing/zxing} (Zugriff am 25.09.2025). — Barcode- und QR-Code-Bibliothek
  \item OWASP Foundation: Cross-Site Request Forgery (CSRF) Prevention Cheat Sheet. 2025. Verfügbar unter: \url{https://cheatsheetseries.owasp.org/cheatsheets/Cross-Site_Request_Forgery_Prevention_Cheat_Sheet.html} (Zugriff am 25.09.2025). — Sicherheitsleitfaden
  \item MDN Web Docs: HTTPS. 2025. Verfügbar unter: \url{https://developer.mozilla.org/docs/Web/HTTP/Overview} (Zugriff am 25.09.2025). — Einführung in HTTP/HTTPS
  \item NIST: Recommendation for Galois/Counter Mode (GCM) for Cryptographic Block Ciphers (SP 800-38D). 2007. Verfügbar unter: \url{https://doi.org/10.6028/NIST.SP.800-38D} (Zugriff am 25.09.2025). — AES-GCM Spezifikation
  \item Microsoft: Visual Studio Code Documentation. 2025. Verfügbar unter: \url{https://code.visualstudio.com/docs} (Zugriff am 25.09.2025). — Entwicklungsumgebung
  \item GitHub: GitHub Docs. 2025. Verfügbar unter: \url{https://docs.github.com/} (Zugriff am 25.09.2025). — Versionsverwaltung \& Plattform
  \item FileZilla Project: FileZilla Client Documentation. 2025. Verfügbar unter: \url{https://wiki.filezilla-project.org/Documentation} (Zugriff am 25.09.2025). — FTP-Client
\end{enumerate}


% Anhang ---------------------------------------------------------------------
\clearpage        % sorgt für sauberes Abschließen
\begingroup
  \let\clearpage\relax
  \let\cleardoublepage\relax
  \appendix
  \pagenumbering{Roman}
  % !TEX root = Projektdokumentation.tex
\section{Anhang}

\subsection{Detaillierte Zeitplanung}
\label{app:Zeitplanung}

\tabelleAnhang{ZeitplanungKomplett}

\subsection{Lastenheft (Auszug)}
\label{app:Lastenheft}
Es folgt ein Auszug aus dem Lastenheft mit Fokus auf die Anforderungen:

Die Anwendung muss folgende Anforderungen erfüllen: 
\begin{enumerate}[itemsep=0em,partopsep=0em,parsep=0em,topsep=0em]
\item Benutzeroberfläche (\ac{GUI})
	\begin{enumerate}
	\item Die Webanwendung muss über ein \ac{GUI} verfügen, über das sich sowohl 
    Werber, als auch Geworbene mit ihrer Email-Adresse
    registrieren können.
	\item Das Design des \ac{GUI} soll schlicht und modern gehalten werden.
        \item Außerdem soll es einen Mitarbeiterbereich mit Login geben,
        in dem Mitarbeiter die Möglichkeit haben den \ac{QR}-Code von Gutscheinen
        zu scannen und die Gültigkeit zu überprüfen, sowie den Gutschein zu entwerten.
	\end{enumerate}
\item Responsives Design
	\begin{enumerate}
	\item Die Webanwendung soll in allen gängigen Browsern funktionieren.
        \item Sie soll auf Monitoren aller Größen über den PC nutzbar sein.
        \item Außerdem soll sie über mobile Geräte nutzbar sein.
	\end{enumerate}
\item Sicherheit und Datenschutz
	\begin{enumerate}
        \item Die Kundendaten (Email-Adresse) sollen in verschlüsselter Form
        in der Datenbank gespeichert werden, um sensible Daten zusätzlich zu 
        schützen
        \item Die Datenübertragungen sollen mit \ac{TLS} gesichert sein.
        \item Die \ac{DSGVO} Konformität soll gewährleistet sein.
	\end{enumerate}
\item Automatische Affiliate-Link / Gutscheinerzeugung und Versand (Backend)
        \begin{enumerate}
            \item Bei Registrierung eines Werbers per Email-Adresse soll dieser 
            automatisch einen Affiliate-Link zugesandt bekommen, welcher mit seiner
            user\_id in der Datenbank assoziiert ist.
            \item Registriert sich ein Geworbener durch Aufruf des Affiliate-Links eines Werbers, soll dem Geworbenen ein Affiliate-Link und ein Gutschein
            im \ac{PDF}-Format, inklusive \ac{QR}-Code zum Einlösen des Gutscheins, 
            automatisch zugesandt werden. 
            \item Wird der Gutschein eines Geworbenen von einem Mitarbeiter durch Scannen des \ac{QR}-Codes auf dem Gutschein eingelöst, soll dem Werber ebenfalls ein Gutschein im \ac{PDF}-Format erzeugt und zugesandt werden.
        \end{enumerate}
\end{enumerate}
\newpage
% \subsection{Pflichtenheft (Auszug)}
\label{app:Pflichtenheft}
Es folgt ein Auszug aus dem Pflichtenheft mit Fokus auf die konkret umzusetzenden Anforderungen:

\begin{enumerate}[itemsep=0em,partopsep=0em,parsep=0em,topsep=0em]
    \item Musskriterien
    \begin{enumerate}
        \item Empfehlungslinks
        \begin{itemize}
            \item Werber können personalisierte Empfehlungslinks generieren.
            \item Empfehlungslinks enthalten einen eindeutigen Token und einen QR-Code.
            \item Geworbene registrieren sich über diesen Link und werden automatisch dem Werber zugeordnet.
        \end{itemize}

        \item Gutscheinsystem
        \begin{itemize}
            \item Ein Mitarbeiter überprüft die Gültigkeit des Gutscheins per App-Scan.
            \item Nach erfolgreicher Prüfung wird der Gutschein entwertet.
            \item Beim Kauf bestimmter Produkte unter Vorlage des Gutscheins erhält der Werber \bzw der Geworbene einen Rabatt in Höhe von \SI{10}{\percent}.
        \end{itemize}

        \item Benutzeroberfläche
        \begin{itemize}
            \item Eine Weboberfläche ermöglicht die Registrierung für Werber und Geworbene.
            \item Ein Mitarbeiterbereich erlaubt nach Login mit einem Pincode das Scannen und Entwerten (nach erfolgreicher automatischer Validierung) von Gutscheinen.
            \item Die Oberfläche soll schlicht, modern und responsiv gestaltet sein.
        \end{itemize}

        \item Datenverwaltung
        \begin{itemize}
            \item Speicherung und Verwaltung der generierten Links und Gutscheincodes in einer Datenbank.
            \item Historisierung relevanter Daten zur Nachvollziehbarkeit.
            \item Absicherung der erfassten personenbezogenen Daten (E-Mail-Adressen).
        \end{itemize}
    \end{enumerate}

    \item Rahmenbedingungen und Abgrenzung
    \begin{enumerate}
        \item Rechtliche Anforderungen an Datenschutz (DSGVO) sind einzuhalten.
        \item Die Lösung soll innerhalb des vorgegebenen Projektzeitraums implementiert und dokumentiert werden.
        \item Es sind nur Standard-Webtechnologien erlaubt, externe Lizenzkosten sollen vermieden werden.
    \end{enumerate}
\end{enumerate}


% \clearpage
\subsection{Use Case-Diagramm}
\label{app:UseCase}
% Use Case-Diagramme und weitere \acs{UML}-Diagramme kann man auch direkt mit \LaTeX{} zeichnen, siehe \zB \url{http://metauml.sourceforge.net/old/usecase-diagram.html}.
\begin{figure}[htb]

\vspace{7pt}
\begin{center}
\begin{tikzpicture}[
  font=\sffamily\itshape\fontsize{13}{15}\selectfont,
  actor/.style={draw, minimum size=1cm},
  include/.style={dashed,->,>=Latex, line width=1pt},
  extend/.style={dashed,->,>=Latex, line width=1pt},
  note/.style={
    draw,
    fill=yellow!20,
    font=\scriptsize,
    align=left,
    inner sep=3pt,
    rectangle,
    minimum width=2.7cm,
    minimum height=1.7cm,
    path picture={
      % gefaltete Ecke oben links (IHK-konform)
      \draw (path picture bounding box.north west) ++(0.4,0)
        -- ++(0,-0.4) -- (path picture bounding box.north west) -- cycle;
    }
  }
]

    % Funktion zum Zeichnen eines Strichmännchens
    \newcommand{\actor}[3]{%
        % Kopf
        \draw (#1,#2) circle (0.3);
        % Körper
        \draw (#1,#2-0.3) -- (#1,#2-1);
        % Arme
        \draw (#1-0.5,#2-0.6) -- (#1+0.5,#2-0.6);
        % Beine
        \draw (#1,#2-1) -- (#1-0.5,#2-1.5);
        \draw (#1,#2-1) -- (#1+0.5,#2-1.5);
        % Name
        \node[below=1.7cm] at (#1,#2) {#3};
    }

    % Systemgrenze
    \draw[thick, fill=yellow!10]
        (-7,-12.1) rectangle (3,5)
        node[above,xshift=-5cm, yshift=-0.8cm, fill=yellow!25] {Empfehlungsmarketing-App};

    % Akteure
    \actor{-8.5}{3}{Werber}
    \actor{-8.5}{0}{Geworbener}
    \actor{4.5}{3}{Mitarbeiter}

    % Use Cases
    \node[draw, ellipse, fill=yellow!50, minimum height=1cm, minimum width=3cm,
    text width=5cm, align=center] (ucLogin) at (-2,3.0) {Einloggen in Mitarbeiterbereich};

    \node[draw, ellipse, fill=yellow!50, minimum height=1cm, minimum width=3cm,
    text width=5cm, align=center] (ucRedeem) at (-2,-1.0) {Gutschein scannen / prüfen / entwerten};

    \node[draw, ellipse, fill=yellow!50, minimum height=1cm, minimum width=3cm,
    text width=2.5cm, align=center] (ucAffLink) at (-4.7,-6.5) {Affiliate-Link senden};

    \node[draw, ellipse, fill=yellow!50, minimum height=1cm, minimum width=3cm,
    text width=2cm, align=center] (ucVoucherSend) at (1.0,-6.5) {Gutschein senden};

    \node[draw, ellipse, fill=yellow!50, minimum height=1cm, minimum width=3cm,
    text width=5cm, align=center] (ucRegister) at (-2,-9.4) {Registrieren mit Email-Adresse};

    % Verbindungen zu Akteuren
    \draw (-7.5,2.0) -- (ucRegister.west);
    \draw (-7.5,-0.5) -- (ucRegister.west);
    \draw (3.5,1.8) -- (ucRedeem.east);
    \draw (3.5,2.2) -- (ucLogin.east);

    % <<include>>: Gutscheinprüfung benötigt Login
    \draw[include] (ucRedeem.north) --
      node[right,sloped,above]{\fontsize{11}{13}\selectfont\bfseries <<include>>}
      (ucLogin.south);

    % <<include>>: Registrierung umfasst immer Affiliate-Link und Gutschein senden
    \draw[include] (ucAffLink.south) --
      node[right,sloped,above]{\fontsize{11}{13}\selectfont\bfseries <<include>>}
      (ucRegister.north);
    \draw[extend] (ucVoucherSend.south) --
      node[left,sloped,above]{\fontsize{11}{13}\selectfont\bfseries <<extend>>}
      (ucRegister.north);

    % <<extend>>: Gutschein einlösen -> Werber erhält Gutschein
    \draw[extend] (ucVoucherSend.north east) -- (ucRedeem.south east)
      node[midway,sloped,above]{\fontsize{11}{13}\selectfont\bfseries <<extend>>};

    % UML-Notiz (IHK-konform, Ecke oben links), an die Beziehung gehängt
    \node[note] (note1) at (-2,-3.0) {Bedingung:\\ Gutschein entwertet};
    \draw[dashed] (note1.west) -- ++(-0.5,0) |- ($(ucVoucherSend.north east)!0.45!(ucRedeem.south east)$);

    \node[note] (note2) at (-2,-5.0) {Bedingung:\\ Registrierung über\\ Affiliate-Link};
    \draw[dashed] (note2.south) -- ++(0,0) |- ($(ucVoucherSend.south)!0.05!(ucRegister.north)$);

    % ===== Legende unten rechts =====
    \node[draw, fill=yellow!20, align=left, font=\scriptsize, anchor=north east] at (3,-10.3) (legend) {
        \textbf{Legende}\\[3pt]
        \begin{tabular}{@{}ll@{}}
          \raisebox{0.5ex}{\tikz{\draw[dashed,->,>=Latex,line width=1pt] (0,0)--(0.8,0);}} & <<include>> = verpflichtender Use Case \\[6pt]
          \raisebox{0.5ex}{\tikz{\draw[dashed,->,>=Latex,line width=1pt] (0,0)--(0.8,0);}} & <<extend>> = optionaler Use Case \\
        \end{tabular}
    };

\end{tikzpicture}
\end{center}

% \includegraphicsKeepAspectRatio{UseCase.pdf}{0.7}
\caption{Use Case-Diagramm}
\end{figure}


\newpage
\subsection{Datenbankmodell}
\label{app:Datenbankmodell}
\begin{figure}[h]
\centering
\begin{tikzpicture}[
    font=\sffamily\small,
    >=Latex,
    node distance=18mm and 24mm,
    entity/.style={
        draw,
        rounded corners,
        fill=blue!5,
        align=left,
        inner sep=3.5mm,
        minimum width=4.4cm,
        text depth=0pt
      },
    rel/.style={-Latex, thick},
    card/.style={midway, fill=white, inner sep=1pt}
  ]

  % --- Entities ---------------------------------------------------
  \node[entity] (users) {\textbf{users}\\
    \pk{id (PK)}\\
    \attr{email\_enc}\\
    \attr{email\_iv}\\
    \attr{email\_tag}\\
    \attr{email\_hash}\\
    \attr{referral\_code}\\
    \fk{referrer\_id (FK)}\\
    \attr{created\_at}
  };

  \node[entity, right=of users] (vouchers) {\textbf{vouchers}\\
    \pk{id (PK)}\\
    \fk{user\_id (FK)}\\
    \attr{code}\\
    \attr{discount\_percent}\\
    \attr{expires\_at}\\
    \attr{created\_at}
  };

  \node[entity, below=of vouchers] (redemptions) {\textbf{redemptions}\\
    \pk{id (PK)}\\
    \fk{voucher\_id (FK)}\\
    \fk{employee\_id (FK)}\\
    \attr{redeemed\_at}
  };

  \node[entity, below=of users] (maillog) {\textbf{mail\_log}\\
    \pk{id (PK)}\\
    \fk{to\_user\_id (FK)}\\
    \attr{subject}\\
    \attr{success}\\
    \attr{error}\\
    \attr{created\_at}
  };

  \node[entity, below=of redemptions] (employees) {\textbf{employees}\\
    \pk{id (PK)}\\
    \attr{first\_name}\\
    \attr{last\_name}\\
    \attr{email}\\
    \attr{created\_at}
  };

  % --- Relationships ----------------------------------------------
  \draw[rel] (users) --
  node[card, xshift=-20pt] {1}
  node[card, pos=0.8, xshift=-2pt] {N}
  (vouchers);

  \draw[rel] (vouchers) --
  node[card, xshift=7pt, yshift=15] {1}
  node[card, pos=0.8, xshift=7pt, yshift=3] {1}
  (redemptions);

  \draw[rel] (employees) --
  node[card, xshift=7pt, yshift=-13] {1}
  node[card, pos=0.8, xshift=7pt, yshift=-1] {N}
  (redemptions);

  \draw[rel] (users) --
  node[card, xshift=7pt, yshift=13] {1}
  node[card, pos=0.8, xshift=7pt, yshift=0] {N}
  (maillog);
\end{tikzpicture}
\caption{Vereinfachtes Datenbankschema (MySQL)}


% \includegraphicsKeepAspectRatio{database.pdf}{1}
% \caption{Datenbankmodell}
\end{figure}

\newpage
\subsection{Screenshots der Anwendung}
\label{sec:screenshots}
\subsubsection{Registrierungsformular}
\begin{figure}[H]
    \centering
    \includegraphics[width=0.8\linewidth]{Bilder/screenshots/form_blank.png}
    \caption{Leeres Formular}
    \label{fig:placeholder}
\end{figure}

\begin{figure}[H]
    \centering
    \includegraphics[width=0.8\linewidth]{Bilder/screenshots/form_registered.png}
    \caption{Erfolgreiche Registrierung}
    \label{fig:placeholder}
\end{figure}

\begin{figure}[H]
    \centering
    \includegraphics[width=0.8\linewidth]{Bilder/screenshots/form_already_registered.png}
    \caption{Fehlermeldung bei doppeltem Registrierungsversuch}
    \label{fig:placeholder}
\end{figure}

\subsubsection{Emails an Werber oder Geworbenen}
\begin{figure}[H]
    \centering
    \includegraphics[width=0.5\linewidth]{Bilder/screenshots/email_voucher.png}
    \caption{Gutschein an Werber/Geworbenen}
    \label{fig:placeholder}
\end{figure}

\begin{figure}[H]
    \centering
    \includegraphics[width=1.0\linewidth]{Bilder/screenshots/mail_affiliate_link.png}
    \caption{Affiliate-Link und QR-Code an Werber/Geworbenen}
    \label{fig:placeholder}
\end{figure}



% \input{Anhang/AnhangEntwuerfe.tex}
% \clearpage
% \input{Anhang/AnhangScreenshots.tex}
% \input{Anhang/AnhangDoc.tex}
% \clearpage
% \input{Anhang/AnhangTest.tex}

% \subsection{Klasse: ComparedNaturalModuleInformation}
% \label{app:CNMI}
% Kommentare und simple Getter/Setter werden nicht angezeigt.
% \lstinputlisting[language=php, caption={Klasse: ComparedNaturalModuleInformation}]{Listings/cnmi.php}
% \clearpage

% \subsection{Klassendiagramm}
% \label{app:Klassendiagramm}
% Klassendiagramme und weitere \acs{UML}-Diagramme kann man auch direkt mit \LaTeX{} zeichnen, siehe \zB \url{http://metauml.sourceforge.net/old/class-diagram.html}.
% \begin{figure}[htb]
% \centering
% \includegraphicsKeepAspectRatio{Klassendiagramm.pdf}{1}
% \caption{Klassendiagramm}
% \end{figure}
% \clearpage

% \input{Anhang/AnhangBenutzerDoku.tex}

\endgroup

\end{document}
