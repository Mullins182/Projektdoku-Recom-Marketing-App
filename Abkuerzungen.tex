% !TEX root = Projektdokumentation.tex

% Es werden nur die Abkürzungen aufgelistet, die mit \ac definiert und auch benutzt wurden. 
%
% \acro{VERSIS}{Versicherungsinformationssystem\acroextra{ (Bestandsführungssystem)}}
% Ergibt in der Liste: VERSIS Versicherungsinformationssystem (Bestandsführungssystem)
% Im Text aber: \ac{VERSIS} -> Versicherungsinformationssystem (VERSIS)

% Hinweis: allgemein bekannte Abkürzungen wie z.B. bzw. u.a. müssen nicht ins Abkürzungsverzeichnis aufgenommen werden
% Hinweis: allgemein bekannte IT-Begriffe wie Datenbank oder Programmiersprache müssen nicht erläutert werden,
%          aber ggfs. Fachbegriffe aus der Domäne des Prüflings (z.B. Versicherung)

% Die Option (in den eckigen Klammern) enthält das längste Label oder
% einen Platzhalter der die Breite der linken Spalte bestimmt.
\begin{acronym}[WWWWW]
	\acro{AES}{Advanced Encryption Standard}\acused{API}
	\acro{API}{Application Programming Interface}\acused{API}
	\acro{CDN}{Content-Delivery-Network}\acused{CDN}
	\acro{Cloud}{Web-Server}\acused{Cloud}
	\acro{CSRF}{Cross-Site Request Forgery}\acused{CSRF}
	\acro{CSS}{Cascading Style Sheets}\acused{CSS}
	\acro{DOM}{Document Object Model}\acused{DOM}
	\acro{DSGVO}{Datenschutzgrundverordnung}\acused{DSGVO}
	\acro{EPK}{Ereignisgesteuerte Prozesskette}\acused{EPK}
	\acro{ERM}{En\-ti\-ty-Re\-la\-tion\-ship-Mo\-dell}
        \acro{FTP}{File Transfer Protocol}\acused{FTP}
        \acro{Git}{verteiltes Versionsverwaltungssystem}\acused{Git}
        \acro{GitHub}{Plattform für Git-Repositories}\acused{GitHub}
        \acro{GUI}{Graphical User Interface}\acused{GUI}
	\acro{HTML}{Hypertext Markup Language}\acused{HTML}
	\acro{PDF}{Portable Document Format}\acused{PDF}
	\acro{PDO}{PhP Data Object}\acused{PDO}
	\acro{PHP}{Hypertext Preprocessor}\acused{PHP}
	\acro{QR}{Quick Response}\acused{QR}
	\acro{SSL/TLS}{Secure Sockets Layer/Transport Layer Security}\acused{SSL/TLS}
	\acro{SQL}{Structured Query Language}\acused{SQL}
	\acro{UI}{User Interface}\acused{UI}
	\acro{UML}{Unified Modeling Language}\acused{UML}
	\acro{VS Code}{Visual Studio Code - Code-Editor von Microsoft}\acused{VS Code}
	\acro{XML}{Extensible Markup Language}\acused{XML}
	\acro{ZXing}{Ausgesprochen: „Zebra Crossing“. Open-Source-Bibliothek zur Erkennung und in einigen Implementierungen auch zur Erstellung von Barcodes und QR-Codes.}\acused{ZXing}
	\acro{@zxing/browser}{Ausgesprochen: Zebra Crossing. Bibliothek zum Scannen von Bar-           und QR-Codes}\acused{@zxing/browser}
\end{acronym}
