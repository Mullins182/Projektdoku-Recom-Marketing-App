% !TEX root = Projektdokumentation.tex

% Es werden nur die Abkürzungen aufgelistet, die mit \ac definiert und auch benutzt wurden. 
%
% \acro{VERSIS}{Versicherungsinformationssystem\acroextra{ (Bestandsführungssystem)}}
% Ergibt in der Liste: VERSIS Versicherungsinformationssystem (Bestandsführungssystem)
% Im Text aber: \ac{VERSIS} -> Versicherungsinformationssystem (VERSIS)

% Hinweis: allgemein bekannte Abkürzungen wie z.B. bzw. u.a. müssen nicht ins Abkürzungsverzeichnis aufgenommen werden
% Hinweis: allgemein bekannte IT-Begriffe wie Datenbank oder Programmiersprache müssen nicht erläutert werden,
%          aber ggfs. Fachbegriffe aus der Domäne des Prüflings (z.B. Versicherung)

% Die Option (in den eckigen Klammern) enthält das längste Label oder
% einen Platzhalter der die Breite der linken Spalte bestimmt.

\begin{acronym}[WWWWW]
  \acro{AES}{Advanced Encryption Standard: Symmetrisches Verschlüsselungsverfahren.}\acused{AES}
  \acro{API}{Application Programming Interface: Schnittstelle zur Anwendungsprogrammierung.}\acused{API}
  \acro{CDN}{Content Delivery Network: Verteiltes Netz von Servern zur schnelleren Bereitstellung von Inhalten.}\acused{CDN}
  \acro{Cloud}{Cloud Computing: Bereitstellung von IT-Ressourcen über das Internet (z.\,B. Speicher oder Server).}\acused{Cloud}
  \acro{Composer}{Composer: Paket- und Abhängigkeitsverwaltung für PHP-Projekte.}\acused{Composer}
  \acro{CSRF}{Cross-Site Request Forgery: Angriffstechnik zur Ausnutzung von Sitzungen im Web.}\acused{CSRF}
  \acro{CSS}{Cascading Style Sheets: Sprache zur Gestaltung und Formatierung von Webseiten.}\acused{CSS}
  \acro{DOM}{Document Object Model: Standardisierte Schnittstelle zur strukturierten Darstellung von HTML/XML-Dokumenten.}\acused{DOM}
  \acro{DSGVO}{Datenschutz-Grundverordnung: EU-Verordnung zum Schutz personenbezogener Daten.}\acused{DSGVO}
  \acro{EPK}{Ereignisgesteuerte Prozesskette: Methode zur Modellierung von Geschäftsprozessen.}\acused{EPK}
  \acro{ERM}{Entity-Relationship-Modell: Datenmodell zur Darstellung von Entitäten und Beziehungen.}\acused{ERM}
  \acro{FTP}{File Transfer Protocol: Protokoll zur Übertragung von Dateien über ein Netzwerk.}\acused{FTP}
  \acro{Git}{Git: Verteiltes Versionsverwaltungssystem für Quellcode.}\acused{Git}
  \acro{GitHub}{GitHub: Plattform zur Verwaltung von Git-Repositories in der Cloud.}\acused{GitHub}
  \acro{GUI}{Graphical User Interface: Grafische Benutzeroberfläche.}\acused{GUI}
  \acro{HTML}{Hypertext Markup Language: Auszeichnungssprache für Webseiten.}\acused{HTML}
  \acro{HTTPS}{Hypertext Transfer Protocol Secure: Verschlüsseltes Übertragungsprotokoll für sichere Kommunikation im Web.}\acused{HTTPS}
  \acro{JSON}{JavaScript Object Notation: Textbasiertes, leichtgewichtiges Datenformat zur strukturierten Darstellung von Informationen, das häufig für den Austausch zwischen Anwendungen und Webschnittstellen genutzt wird.}\acused{JSON}
  \acro{MySQL}{MySQL: Relationales Open-Source-Datenbankmanagementsystem (RDBMS).}\acused{MySQL}
  \acro{PDF}{Portable Document Format: Plattformunabhängiges Dateiformat für Dokumente.}\acused{PDF}
  \acro{PDO}{PHP Data Objects: Datenbankschnittstelle in PHP für verschiedene Datenbanken.}\acused{PDO}
  \acro{PHP}{Hypertext Preprocessor: Serverseitige Skriptsprache für Webanwendungen.}\acused{PHP}
  \acro{QR}{Quick Response Code: Zweidimensionaler Barcode zur Codierung von Daten.}\acused{QR}
  \acro{TLS}{Secure Sockets Layer / Transport Layer Security: Verschlüsselungsprotokolle für sichere Datenübertragung.}\acused{TLS}
  \acro{SQL}{Structured Query Language: Sprache zur Abfrage und Manipulation relationaler Datenbanken.}\acused{SQL}
  \acro{UI}{User Interface: Benutzerschnittstelle zwischen Mensch und System.}\acused{UI}
  \acro{UML}{Unified Modeling Language: Standardisierte Modellierungssprache für Softwarearchitektur.}\acused{UML}
  \acro{URL}{Uniform Resource Locator: Eindeutige Adresse einer Ressource im Internet.}\acused{URL}
  \acro{VS Code}{Visual Studio Code: Quelloffene Entwicklungsumgebung von Microsoft.}\acused{VS Code}
  \acro{XML}{Extensible Markup Language: Auszeichnungssprache zur strukturierten Speicherung von Daten.}\acused{XML}
  \acro{ZXing}{ZXing („Zebra Crossing“): Open-Source-Bibliothek zur Erkennung und Erstellung von Barcodes und QR-Codes.}\acused{ZXing}
  \acro{@zxing/browser}{@zxing/browser: JavaScript-Bibliothek aus dem ZXing-Projekt zum Scannen von Bar- und QR-Codes im Browser.}\acused{@zxing/browser}
\end{acronym}
