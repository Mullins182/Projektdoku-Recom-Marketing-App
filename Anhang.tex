% !TEX root = Projektdokumentation.tex
\section{Anhang}

\subsection{Detaillierte Zeitplanung}
\label{app:Zeitplanung}

\tabelleAnhang{ZeitplanungKomplett}

\subsection{Lastenheft (Auszug)}
\label{app:Lastenheft}
Es folgt ein Auszug aus dem Lastenheft mit Fokus auf die Anforderungen:

Die Anwendung muss folgende Anforderungen erfüllen: 
\begin{enumerate}[itemsep=0em,partopsep=0em,parsep=0em,topsep=0em]
\item Benutzeroberfläche (\ac{GUI})
	\begin{enumerate}
	\item Die Webanwendung muss über ein \ac{GUI} verfügen, über das sich sowohl 
    Werber, als auch Geworbene mit ihrer Email-Adresse
    registrieren können.
	\item Das Design des \ac{GUI} soll schlicht und modern gehalten werden.
        \item Außerdem soll es einen Mitarbeiterbereich mit Login geben,
        in dem Mitarbeiter die Möglichkeit haben den \ac{QR}-Code von Gutscheinen
        zu scannen und die Gültigkeit zu überprüfen, sowie den Gutschein zu entwerten.
	\end{enumerate}
\item Responsives Design
	\begin{enumerate}
	\item Die Webanwendung soll in allen gängigen Browsern funktionieren.
        \item Sie soll auf Monitoren aller Größen über den PC nutzbar sein.
        \item Außerdem soll sie über mobile Geräte nutzbar sein.
	\end{enumerate}
\item Sicherheit und Datenschutz
	\begin{enumerate}
        \item Die Kundendaten (Email-Adresse) sollen in verschlüsselter Form
        in der Datenbank gespeichert werden, um sensible Daten zusätzlich zu 
        schützen
        \item Die Datenübertragungen sollen mit \ac{TLS} gesichert sein.
        \item Die \ac{DSGVO} Konformität soll gewährleistet sein.
	\end{enumerate}
\item Automatische Affiliate-Link / Gutscheinerzeugung und Versand (Backend)
        \begin{enumerate}
            \item Bei Registrierung eines Werbers per Email-Adresse soll dieser 
            automatisch einen Affiliate-Link zugesandt bekommen, welcher mit seiner
            user\_id in der Datenbank assoziiert ist.
            \item Registriert sich ein Geworbener durch Aufruf des Affiliate-Links eines Werbers, soll dem Geworbenen ein Affiliate-Link und ein Gutschein
            im \ac{PDF}-Format, inklusive \ac{QR}-Code zum Einlösen des Gutscheins, 
            automatisch zugesandt werden. 
            \item Wird der Gutschein eines Geworbenen von einem Mitarbeiter durch Scannen des \ac{QR}-Codes auf dem Gutschein eingelöst, soll dem Werber ebenfalls ein Gutschein im \ac{PDF}-Format erzeugt und zugesandt werden.
        \end{enumerate}
\end{enumerate}
\newpage
% \subsection{Pflichtenheft (Auszug)}
\label{app:Pflichtenheft}
Es folgt ein Auszug aus dem Pflichtenheft mit Fokus auf die konkret umzusetzenden Anforderungen:

\begin{enumerate}[itemsep=0em,partopsep=0em,parsep=0em,topsep=0em]
    \item Musskriterien
    \begin{enumerate}
        \item Empfehlungslinks
        \begin{itemize}
            \item Werber können personalisierte Empfehlungslinks generieren.
            \item Empfehlungslinks enthalten einen eindeutigen Token und einen QR-Code.
            \item Geworbene registrieren sich über diesen Link und werden automatisch dem Werber zugeordnet.
        \end{itemize}

        \item Gutscheinsystem
        \begin{itemize}
            \item Ein Mitarbeiter überprüft die Gültigkeit des Gutscheins per App-Scan.
            \item Nach erfolgreicher Prüfung wird der Gutschein entwertet.
            \item Beim Kauf bestimmter Produkte unter Vorlage des Gutscheins erhält der Werber \bzw der Geworbene einen Rabatt in Höhe von \SI{10}{\percent}.
        \end{itemize}

        \item Benutzeroberfläche
        \begin{itemize}
            \item Eine Weboberfläche ermöglicht die Registrierung für Werber und Geworbene.
            \item Ein Mitarbeiterbereich erlaubt nach Login mit einem Pincode das Scannen und Entwerten (nach erfolgreicher automatischer Validierung) von Gutscheinen.
            \item Die Oberfläche soll schlicht, modern und responsiv gestaltet sein.
        \end{itemize}

        \item Datenverwaltung
        \begin{itemize}
            \item Speicherung und Verwaltung der generierten Links und Gutscheincodes in einer Datenbank.
            \item Historisierung relevanter Daten zur Nachvollziehbarkeit.
            \item Absicherung der erfassten personenbezogenen Daten (E-Mail-Adressen).
        \end{itemize}
    \end{enumerate}

    \item Rahmenbedingungen und Abgrenzung
    \begin{enumerate}
        \item Rechtliche Anforderungen an Datenschutz (DSGVO) sind einzuhalten.
        \item Die Lösung soll innerhalb des vorgegebenen Projektzeitraums implementiert und dokumentiert werden.
        \item Es sind nur Standard-Webtechnologien erlaubt, externe Lizenzkosten sollen vermieden werden.
    \end{enumerate}
\end{enumerate}


% \clearpage
\subsection{Use Case-Diagramm}
\label{app:UseCase}
% Use Case-Diagramme und weitere \acs{UML}-Diagramme kann man auch direkt mit \LaTeX{} zeichnen, siehe \zB \url{http://metauml.sourceforge.net/old/usecase-diagram.html}.
\begin{figure}[htb]

\vspace{7pt}
\begin{center}
\begin{tikzpicture}[
  font=\sffamily\itshape\fontsize{13}{15}\selectfont,
  actor/.style={draw, minimum size=1cm},
  include/.style={dashed,->,>=Latex, line width=1pt},
  extend/.style={dashed,->,>=Latex, line width=1pt},
  note/.style={
    draw,
    fill=yellow!20,
    font=\scriptsize,
    align=left,
    inner sep=3pt,
    rectangle,
    minimum width=2.7cm,
    minimum height=1.7cm,
    path picture={
      % gefaltete Ecke oben links (IHK-konform)
      \draw (path picture bounding box.north west) ++(0.4,0)
        -- ++(0,-0.4) -- (path picture bounding box.north west) -- cycle;
    }
  }
]

    % Funktion zum Zeichnen eines Strichmännchens
    \newcommand{\actor}[3]{%
        % Kopf
        \draw (#1,#2) circle (0.3);
        % Körper
        \draw (#1,#2-0.3) -- (#1,#2-1);
        % Arme
        \draw (#1-0.5,#2-0.6) -- (#1+0.5,#2-0.6);
        % Beine
        \draw (#1,#2-1) -- (#1-0.5,#2-1.5);
        \draw (#1,#2-1) -- (#1+0.5,#2-1.5);
        % Name
        \node[below=1.7cm] at (#1,#2) {#3};
    }

    % Systemgrenze
    \draw[thick, fill=yellow!10]
        (-7,-12.1) rectangle (3,5)
        node[above,xshift=-5cm, yshift=-0.8cm, fill=yellow!25] {Empfehlungsmarketing-App};

    % Akteure
    \actor{-8.5}{3}{Werber}
    \actor{-8.5}{0}{Geworbener}
    \actor{4.5}{3}{Mitarbeiter}

    % Use Cases
    \node[draw, ellipse, fill=yellow!50, minimum height=1cm, minimum width=3cm,
    text width=5cm, align=center] (ucLogin) at (-2,3.0) {Einloggen in Mitarbeiterbereich};

    \node[draw, ellipse, fill=yellow!50, minimum height=1cm, minimum width=3cm,
    text width=5cm, align=center] (ucRedeem) at (-2,-1.0) {Gutschein scannen / prüfen / entwerten};

    \node[draw, ellipse, fill=yellow!50, minimum height=1cm, minimum width=3cm,
    text width=2.5cm, align=center] (ucAffLink) at (-4.7,-6.5) {Affiliate-Link senden};

    \node[draw, ellipse, fill=yellow!50, minimum height=1cm, minimum width=3cm,
    text width=2cm, align=center] (ucVoucherSend) at (1.0,-6.5) {Gutschein senden};

    \node[draw, ellipse, fill=yellow!50, minimum height=1cm, minimum width=3cm,
    text width=5cm, align=center] (ucRegister) at (-2,-9.4) {Registrieren mit Email-Adresse};

    % Verbindungen zu Akteuren
    \draw (-7.5,2.0) -- (ucRegister.west);
    \draw (-7.5,-0.5) -- (ucRegister.west);
    \draw (3.5,1.8) -- (ucRedeem.east);
    \draw (3.5,2.2) -- (ucLogin.east);

    % <<include>>: Gutscheinprüfung benötigt Login
    \draw[include] (ucRedeem.north) --
      node[right,sloped,above]{\fontsize{11}{13}\selectfont\bfseries <<include>>}
      (ucLogin.south);

    % <<include>>: Registrierung umfasst immer Affiliate-Link und Gutschein senden
    \draw[include] (ucAffLink.south) --
      node[right,sloped,above]{\fontsize{11}{13}\selectfont\bfseries <<include>>}
      (ucRegister.north);
    \draw[extend] (ucVoucherSend.south) --
      node[left,sloped,above]{\fontsize{11}{13}\selectfont\bfseries <<extend>>}
      (ucRegister.north);

    % <<extend>>: Gutschein einlösen -> Werber erhält Gutschein
    \draw[extend] (ucVoucherSend.north east) -- (ucRedeem.south east)
      node[midway,sloped,above]{\fontsize{11}{13}\selectfont\bfseries <<extend>>};

    % UML-Notiz (IHK-konform, Ecke oben links), an die Beziehung gehängt
    \node[note] (note1) at (-2,-3.0) {Bedingung:\\ Gutschein entwertet};
    \draw[dashed] (note1.west) -- ++(-0.5,0) |- ($(ucVoucherSend.north east)!0.45!(ucRedeem.south east)$);

    \node[note] (note2) at (-2,-5.0) {Bedingung:\\ Registrierung über\\ Affiliate-Link};
    \draw[dashed] (note2.south) -- ++(0,0) |- ($(ucVoucherSend.south)!0.05!(ucRegister.north)$);

    % ===== Legende unten rechts =====
    \node[draw, fill=yellow!20, align=left, font=\scriptsize, anchor=north east] at (3,-10.3) (legend) {
        \textbf{Legende}\\[3pt]
        \begin{tabular}{@{}ll@{}}
          \raisebox{0.5ex}{\tikz{\draw[dashed,->,>=Latex,line width=1pt] (0,0)--(0.8,0);}} & <<include>> = verpflichtender Use Case \\[6pt]
          \raisebox{0.5ex}{\tikz{\draw[dashed,->,>=Latex,line width=1pt] (0,0)--(0.8,0);}} & <<extend>> = optionaler Use Case \\
        \end{tabular}
    };

\end{tikzpicture}
\end{center}

% \includegraphicsKeepAspectRatio{UseCase.pdf}{0.7}
\caption{Use Case-Diagramm}
\end{figure}


\newpage
\subsection{Datenbankmodell}
\label{app:Datenbankmodell}
\begin{figure}[h]
\centering
\begin{tikzpicture}[
    font=\sffamily\small,
    >=Latex,
    node distance=18mm and 24mm,
    entity/.style={
        draw,
        rounded corners,
        fill=blue!5,
        align=left,
        inner sep=3.5mm,
        minimum width=4.4cm,
        text depth=0pt
      },
    rel/.style={-Latex, thick},
    card/.style={midway, fill=white, inner sep=1pt}
  ]

  % --- Entities ---------------------------------------------------
  \node[entity] (users) {\textbf{users}\\
    \pk{id (PK)}\\
    \attr{email\_enc}\\
    \attr{email\_iv}\\
    \attr{email\_tag}\\
    \attr{email\_hash}\\
    \attr{referral\_code}\\
    \fk{referrer\_id (FK)}\\
    \attr{created\_at}
  };

  \node[entity, right=of users] (vouchers) {\textbf{vouchers}\\
    \pk{id (PK)}\\
    \fk{user\_id (FK)}\\
    \attr{code}\\
    \attr{discount\_percent}\\
    \attr{expires\_at}\\
    \attr{created\_at}
  };

  \node[entity, below=of vouchers] (redemptions) {\textbf{redemptions}\\
    \pk{id (PK)}\\
    \fk{voucher\_id (FK)}\\
    \fk{employee\_id (FK)}\\
    \attr{redeemed\_at}
  };

  \node[entity, below=of users] (maillog) {\textbf{mail\_log}\\
    \pk{id (PK)}\\
    \fk{to\_user\_id (FK)}\\
    \attr{subject}\\
    \attr{success}\\
    \attr{error}\\
    \attr{created\_at}
  };

  \node[entity, below=of redemptions] (employees) {\textbf{employees}\\
    \pk{id (PK)}\\
    \attr{first\_name}\\
    \attr{last\_name}\\
    \attr{email}\\
    \attr{created\_at}
  };

  % --- Relationships ----------------------------------------------
  \draw[rel] (users) --
  node[card, xshift=-20pt] {1}
  node[card, pos=0.8, xshift=-2pt] {N}
  (vouchers);

  \draw[rel] (vouchers) --
  node[card, xshift=7pt, yshift=15] {1}
  node[card, pos=0.8, xshift=7pt, yshift=3] {1}
  (redemptions);

  \draw[rel] (employees) --
  node[card, xshift=7pt, yshift=-13] {1}
  node[card, pos=0.8, xshift=7pt, yshift=-1] {N}
  (redemptions);

  \draw[rel] (users) --
  node[card, xshift=7pt, yshift=13] {1}
  node[card, pos=0.8, xshift=7pt, yshift=0] {N}
  (maillog);
\end{tikzpicture}
\caption{Vereinfachtes Datenbankschema (MySQL)}


% \includegraphicsKeepAspectRatio{database.pdf}{1}
% \caption{Datenbankmodell}
\end{figure}

\newpage
\subsection{Screenshots der Anwendung}
\label{sec:screenshots}
\subsubsection{Registrierungsformular}
\begin{figure}[H]
    \centering
    \includegraphics[width=0.8\linewidth]{Bilder/screenshots/form_blank.png}
    \caption{Leeres Formular}
    \label{fig:placeholder}
\end{figure}

\begin{figure}[H]
    \centering
    \includegraphics[width=0.8\linewidth]{Bilder/screenshots/form_registered.png}
    \caption{Erfolgreiche Registrierung}
    \label{fig:placeholder}
\end{figure}

\begin{figure}[H]
    \centering
    \includegraphics[width=0.8\linewidth]{Bilder/screenshots/form_already_registered.png}
    \caption{Fehlermeldung bei doppeltem Registrierungsversuch}
    \label{fig:placeholder}
\end{figure}

\subsubsection{Emails an Werber oder Geworbenen}
\begin{figure}[H]
    \centering
    \includegraphics[width=0.5\linewidth]{Bilder/screenshots/email_voucher.png}
    \caption{Gutschein an Werber/Geworbenen}
    \label{fig:placeholder}
\end{figure}

\begin{figure}[H]
    \centering
    \includegraphics[width=1.0\linewidth]{Bilder/screenshots/mail_affiliate_link.png}
    \caption{Affiliate-Link und QR-Code an Werber/Geworbenen}
    \label{fig:placeholder}
\end{figure}



% \input{Anhang/AnhangEntwuerfe.tex}
% \clearpage
% \input{Anhang/AnhangScreenshots.tex}
% \input{Anhang/AnhangDoc.tex}
% \clearpage
% \input{Anhang/AnhangTest.tex}

% \subsection{Klasse: ComparedNaturalModuleInformation}
% \label{app:CNMI}
% Kommentare und simple Getter/Setter werden nicht angezeigt.
% \lstinputlisting[language=php, caption={Klasse: ComparedNaturalModuleInformation}]{Listings/cnmi.php}
% \clearpage

% \subsection{Klassendiagramm}
% \label{app:Klassendiagramm}
% Klassendiagramme und weitere \acs{UML}-Diagramme kann man auch direkt mit \LaTeX{} zeichnen, siehe \zB \url{http://metauml.sourceforge.net/old/class-diagram.html}.
% \begin{figure}[htb]
% \centering
% \includegraphicsKeepAspectRatio{Klassendiagramm.pdf}{1}
% \caption{Klassendiagramm}
% \end{figure}
% \clearpage

% \input{Anhang/AnhangBenutzerDoku.tex}
